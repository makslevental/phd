\usepackage{catchfilebetweentags}
\newcommand{\loadeqn}[1]{%
    \ExecuteMetaData[equations.tex]{eq#1}%
}
\newcommand{\loadfig}[1]{%
    \ExecuteMetaData[figures.tex]{fig#1}%
}
\usepackage[margin=0.7in]{geometry}
\usepackage[parfill]{parskip}
\usepackage[backend=biber, bibencoding=utf8, style=ieee, backref=true]{biblatex}
\usepackage[hyperfootnotes=false,hyperindex=true]{hyperref}
\usepackage[utf8]{inputenc}
\usepackage{amsmath,amssymb,amsfonts,amsthm} % math
\usepackage{bm} % \bm 
\usepackage{physics} % \pdv
\usepackage{mathtools} % coloneqq
\usepackage{imakeidx} % index
\usepackage{etoc} % localtabelofcontents
\usepackage{tikz}
\usetikzlibrary{
    decorations.markings % contour integral
}
\usepackage{sepfootnotes}
\newfootnotes{a}

\usepackage{pgfplots}
\usepackage{subcaption}
\DeclareCaptionFormat{subfig}{#1#2#3}
\DeclareCaptionSubType{figure}
\captionsetup[subfigure]{format=subfig,labelsep=space,labelformat=parens}


\newcommand{\newterm}[1]{\textit{#1}\index{#1}}
\newcommand{\Xk}{X_k}
\newcommand{\Xkone}{X_{k+1}}
\newcommand{\bx}{\bm{x}}
\newcommand{\bX}{\bm{X}}
\newcommand{\bxi}{\bm{x}_i}
\newcommand{\delx}{\bx - \bxi}
\newcommand{\by}{\bm{y}}
\newcommand{\bY}{\bm{Y}}
\newcommand{\byi}{\bm{y}_i}
\newcommand{\dely}{\by - \byi}
\newcommand{\zbx}{Z(\bx)}
\newcommand{\zbxi}{Z(\bxi)}
\newcommand{\bb}{\bm{\beta}}
\newcommand{\hzbx}{\hat{Z}(\bx)}
\NewDocumentCommand{\evalat}{sO{\big}mm}{%
  \IfBooleanTF{#1}
   {\mleft. #3 \mright|_{#4}}
   {#3#2|_{#4}}%
}


\makeindex


% other packages
\usepackage{graphicx}
\usepackage{enumitem}
\usepackage{color}
\usepackage{hyperref}
\hypersetup{
    colorlinks=true,
    linktoc=all,     %set to all if you want both sections and subsections linked
    linkcolor=blue,
}

\makeatletter
\def\thm@space@setup{%
  \thm@preskip=5pt
  \thm@postskip=\thm@preskip % or whatever, if you don't want them to be equal
}
\makeatother

% proper inline math display, adjust height for symbols like \sum
\everymath{\displaystyle}

\theoremstyle{definition}
\newtheorem{example}{Example}[section]

% define tags for math use..
\theoremstyle{plain}% default
\newtheorem{theorem}{Theorem}[section]
\newtheorem{corollary}{Corollary}[theorem]

\theoremstyle{definition}
\newtheorem{defn}{Definition}[section]
\newtheorem{proposition}{Proposition}[defn]
\newtheorem{exmp}{Example}[section]

\theoremstyle{remark}
\newtheorem*{rem}{Remark}
\newtheorem*{note}{Note}
\newtheorem{case}{Case}

% Gives begin{solution} same formating as \begin{proof}
\newenvironment{solution}
  {\begin{proof}[Solution]}
  {\end{proof}}


%running fraction with slash - requires math mode.
\newcommand*\rfrac[2]{{}^{#1}\!/_{#2}}
%shortcut to mathbb
\newcommand{\N}{\mathbb{N}}
\newcommand{\R}{\mathbb{R}}
\newcommand{\I}{\mathbb{I}}
% color highlighting
\newcommand{\hilight}[1]{\colorbox{yellow}{#1}}

\newcommand{\icol}[1]{% inline column vector
  \left(\begin{smallmatrix}#1\end{smallmatrix}\right)%
}

\newcommand{\irow}[1]{% inline row vector
  \begin{smallmatrix}(#1)\end{smallmatrix}%
}
