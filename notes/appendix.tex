\section{Appendix}\label{sec:appendix}
\localtableofcontents

\subsection{Definitions}

\subsubsection{Linear Operator}\label{subsubsec:linearoperator}

A map \(L \colon V \rightarrow W\) between vector spaces over a field \(K\) is a \newterm{linear operator} if
%
\begin{enumerate}
    \item \textbf{distributivity}: \(L(u+v) = L(u) + L(v)\)
    \item \textbf{homogeneity}: \(L(rv) = r L(v)\)
\end{enumerate}
%
To emphasize the field, \(L\) is said to be \(K-\)linear.

\subsubsection{Germs}

Consider the set of all pairs \((f, U)\), where \(U\) is a neighborhood of \(p\) and \(f\colon U \rightarrow \mathbb{R}\) is a \(C^\infty\) function.
%
We say that \((f, U) \sim (g, U')\) if there is an open \(W\) such that \(p \in W \subset U \cap U'\) and \(f = g\) when restricted to \(W\).
%
The equivalence class \([(f, U)]\) of \((f, U)\) is the \newterm{germ} of \(f\) at \(p\).
%
We write
%
\begin{equation}
    C_p^\infty(\mathbb{R}^n) \coloneqq \{[(f, U)]\}
\end{equation}
%
for the set all germs of \(C^\infty\) functions on \(\mathbb{R}^n\) at \(p\).

\subsubsection{Algebra}\label{subsubsec:algebra}

An \newterm{algebra} over \newterm{field} \(K\) is a vector space \(A\) over \(K\) with a multiplication map
%
\begin{equation}
    \mu\colon A \times A \rightarrow A
\end{equation}
%
usually written \(\mu(a,b)=a \cdot b\), such that \(\mu\) is associative, distributive, and homogeneous, where homogeneity is defined:
%
\begin{enumerate}
    \item \textbf{associativity}: \((a\cdot b)\cdot c = a \cdot (b \cdot c) \)
    \item \textbf{distributivity}: \((a+b)\cdot c = a\cdot c + b \cdot c\) and \(a\cdot b + a \cdot c\)
    \item \textbf{homogeneity}: \(r(a\cdot b) = (ra)\cdot b = a\cdot (rb)\)
\end{enumerate}
%
If \(A, A'\) are algebras then an \newterm{algebra homomorphism} is a linear operator \(L\) that respects algebra multiplication \(L(ab) = L(a)L(b)\).
%
It's the case that addition and multiplication of functions \textbf{induces addition and multiplication on the set of germs} \(C_p^\infty\), making it into an algebra over \(\mathbb{R}^n\).

\subsubsection{Module}

If \(R\) is a commutative ring with identity, then a (left) \(R-\)\newterm{module} is an abelian group \(A\) with a scalar multiplication map
%
\begin{equation}
    \mu \colon R\times A \rightarrow A
\end{equation}
%
such that \(\mu\) is
%
\begin{enumerate}
    \item \textbf{associative}: \((rs)a = r (sa)\) for \(r,s \in R\)
    \item \textbf{identity}: \(1 \in R \implies 1a=a\)
    \item \textbf{distributive}: \((r+s)a = ra + sa\) and \(r(a+b) = ra + rb\)
\end{enumerate}
%
If \(R\) is a field, then an \(R-\)module is a vector space over \(R\); in this sense modules generalize vector space to scalars from a ring rather than a field.

Let \(A, A'\) be \(R-\)modules. An \newterm{\(R-\)module homomorphism} \(f\colon A \rightarrow A'\) is a map that preserves both addition and scalar multiplication.


\subsection{Tensor Product}

Let \(f\) be \(k\)-linear function and \(g\) be an \(\ell\)-linear function on a vector space \(V\). Then, their \newterm{tensor product} is the \((k+\ell)\)-linear function \(f\otimes g\)
\begin{equation}
    (f \otimes g) (v_1, \dots, v_{k+\ell}) \coloneqq f(v_1, \dots, v_k)g(v_{k+1}, \dots, v_{k+\ell})
\end{equation}

\begin{example}
    (\newterm{Bilinear maps}) Let \(\{e_i\}\) be a basis for a vector space \(V\) and \(\{\alpha^j\}\) be the dual basis for \(V^\vee\). Also let \(\langle\,,\rangle\colon V \times V \rightarrow \R\) be a bilinear map on \(V\). Set \(g_{ij} = \langle e_i, e_j \rangle\in \R\). If
    \begin{align}
        v & = \sum v^i e_i \\
        w & = \sum w^i e_i
    \end{align}
    then \(v^i = \alpha^i(v)\) and \(w^j = \alpha^j(w)\), where \(\alpha\) are the coordinate functions \(a^i, a^j\). By bilinearity, we can express \(\langle\,,\rangle\) in terms of the tensor product
    \begin{equation}
        \begin{split}
            \langle v,w \rangle &= \sum_{ij} v^i w^j \langle e_i, e_j\rangle \\
            &= \sum \alpha^i(v) \alpha^i(w) g_{ij} \\
            &= \sum (\alpha^i \otimes \alpha^j)(v,w) \times g_{ij}
        \end{split}
    \end{equation}
    Hence \(\langle\,,\rangle = \sum g_{ij}\, \alpha^i \otimes \alpha^j\)
\end{example}

\subsection{Wedge Product}

Let \(f\) be \(k\)-linear function and \(g\) be an \(\ell\)-linear function on a vector space \(V\).
If  \(f,g\) are alternating\anote{alternating} then we would like their product to be alternating as well: the \newterm{wedge product} or \newterm{exterior product} \(f \wedge g\)
\begin{align}
    f \wedge g & \coloneqq \frac{1}{k!l!}  A(f \otimes g) \\
               & \coloneqq 
               \begin{multlined}[t]
                    \frac{1}{k!l!}  \sum_{\sigma \in S_{k+\ell}} (\operatorname{sgn}(\sigma)) f(v_{\sigma(1)}, \dots, v_{\sigma(k)}) \\ 
                    g(v_{\sigma(k+1)}, \dots, v_{\sigma(k+\ell)})
                \end{multlined}
\end{align}
where \(S_{k+\ell}\) is the \newterm{permutation group} on \(k+\ell\) elements.

Note that the wedge product of three alternating functions \(f,g,h\) generalizes to 
\begin{equation}
    f \wedge g \wedge h = \frac{1}{k!\ell!m!} = A(f\otimes g \otimes h)
\end{equation}
and any number of alternating functions.

\begin{proposition}
    (\textit{Wedge product of 1-covectors}) If \(\{\alpha^i\}\) are linear functions on \(V\) and \(v_i \in V\) then 
    \begin{equation}
        \alpha^1 \wedge \cdots \wedge \alpha^k (v_1, \dots, v_k) = \det([\alpha^i(v_j)])
    \end{equation}
    where \([\alpha^i(v_j)]\) is the matrix where the \(i,j\)-entry is \(\alpha^i(v_j)\).
\end{proposition}

\begin{proof}
    \begin{align}
        &\begin{multlined}
        \alpha^1 \wedge \cdots \wedge \alpha^k (v_1, \dots, v_k) = \\  A(\alpha^1 \wedge \cdots \wedge \alpha^k) (v_1, \dots, v_k)
        \end{multlined} \\
        &= \sum_{\sigma \in S_k} (\operatorname{sgn}(\sigma)) \alpha^1 (v_{\sigma(1)}) \wedge \cdots \wedge \alpha^k (v_{\sigma(k)}) \\ 
        &= \det([\alpha^i(v_j)])
    \end{align}
\end{proof}