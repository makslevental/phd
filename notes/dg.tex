

% \subsubsection{Pushforwards and Pullbacks}\label{subsubsec:diffgeo}

% Define a diffeomorphism \(\phi\) such that
% %
% \begin{equation}
%     \phi\colon (r, \theta) \mapsto (r \cos \theta, r \sin \theta)
% \end{equation}
% %
% The Jacobian of \(\phi\)
% %
% \begin{equation}
%     \operatorname{J}
%     =
%     \begin{bmatrix}
%         \pdv{\phi_1}{r} & \pdv{\phi_1}{\theta} \\
%         \pdv{\phi_2}{r} & \pdv{\phi_2}{\theta}
%     \end{bmatrix}
%     =
%     \begin{bmatrix}
%         \cos \theta & -r \sin \theta \\
%         \sin \theta & r \cos \theta
%     \end{bmatrix}
% \end{equation}
% %
% Note that \(\det \operatorname{J} = r\) and so \(\phi\) is a diffeomorphism iff \(r \neq 0\).
% %
% Given a vector field
% %
% \begin{equation}
%     v = a(r, \theta) \partial_r + b(r, \theta)\partial_\theta \coloneqq a(r, \theta) \pdv{}{r} + b(r, \theta)\pdv{}{\theta}
% \end{equation}
% %
% we can compute the \newterm{pushforward} \(\phi_*\) wrt the \(\partial_x, \partial_y\) basis
% \begin{equation}
%     \phi_* (v)
%     =
%     \begin{bmatrix}
%         \cos (\theta) & -r \sin (\theta) \\
%         \sin (\theta) & r \cos (\theta)
%     \end{bmatrix}
%     \cdot
%     \begin{pmatrix}
%         a \\ b
%     \end{pmatrix}
%     =
%     \begin{pmatrix}
%         a \cos (\theta) - br \sin (\theta) \\
%         a\sin (\theta) + br \cos (\theta)
%     \end{pmatrix}
% \end{equation}
% Hence, explicitly
% \begin{equation}
%     \phi_* (v) = (a \cos (\theta) - br \sin (\theta))\partial_x + (a\sin (\theta) + br \cos (\theta))\partial_y
% \end{equation}

% Since \(\operatorname{J}\) is invertible we can investigate which vector fields map to \(\partial_x\)
% %
% \begin{equation}
%     \phi_* v = \partial_x  \iff v = \phi_{*}^{-1} \partial_x
% \end{equation}
% %
% Let \(v = a \partial_r + b \partial_\theta\).
% %
% Then
% \begin{equation}
%     v
%     =
%     \begin{bmatrix}
%         \cos (\theta)            & \sin (\theta)           \\
%         -\frac{\sin (\theta)}{r} & \frac{\cos (\theta)}{r} \\
%     \end{bmatrix}
%     \cdot
%     \begin{pmatrix}
%         1 \\ 0
%     \end{pmatrix} \\
%     =
%     \begin{pmatrix}
%         \cos(\theta) \\ -\frac{\sin (\theta)}{r}
%     \end{pmatrix}
% \end{equation}
% However we need to write \(r, \theta\) in terms of \(x, y\)
% \begin{equation}
%      \phi_{*}^{-1} \partial_x = \frac{x}{\sqrt{x^2+y^2}}\partial_r + \frac{y}{x^2 + y^2} \partial_\theta
% \end{equation}
% %
% \(\phi_{*}^{-1}\) is called the \newterm{pullback} \(\phi^*\) of the vector field \(\partial_x\) along \(\phi\).

\section{Differential Geometry}\label{sec:dg}
\localtableofcontents

\subsection{Directional Derivative}

Elements of the \newterm{tangent space} \(T_p (\mathbb{R}^n)\) anchored at a point \(p = (p^1, \dots, p^n) \in \mathbb{R}^n\) can be visualized as arrows emanating from \(p\).
%
These arrows are called \newterm{tangent vectors} and represented by column vectors:
%
\begin{equation}
    \bm{v}
    =
    \begin{bmatrix}
        v^1 \\ \vdots \\ v^n
    \end{bmatrix}
    % =
    % \begin{bmatrix}
    %     v^1, \dots, v^n    
    % \end{bmatrix}
\end{equation}
%
The line through a point \(p\) with direction \(\bm{v}\) has parameterization
%
\begin{equation}
    c(t) = \left( p^1 + t v^1, \dots, p^n + t v^n \right)
\end{equation}
%
If \(f \in C^\infty\) in a neighborhood of \(p\) and \(\bm{v}\) is a tangent vector at \(p\), the \newterm{directional derivative} of \(f\) in the direction of \(\bm{v}\) at \(p\) is defined
%
\begin{equation}
    D_{\bm{v}} f
    =
    \lim_{t\rightarrow 0}\, \frac{f(c(t)) - f(p)}{t}
    =
    \evalat[\bigg]{\dv{}{t} f(c(t))}{t=0}
\end{equation}
%
By the chain rule
%
\begin{align}
    D_{\bm{v}} f & = \sum_{i=1}^{n} \evalat[\bigg]{\pdv{f}{x^i}}{p} \evalat[\bigg]{\dv{c^i}{t}}{t=0} \\
                 & = \sum_{i=1}^{n} \evalat[\bigg]{\dv{c^i}{t}}{t=0} \evalat[\bigg]{\pdv{f}{x^i}}{p} \\
                 & = \sum_{i=1}^{n} v^i \evalat[\bigg]{\pdv{f}{x^i}}{p}\label{eqn:chainrule}         \\
\end{align}
%
The directional derivative operator at \(p\) is defined
%
\begin{equation}
    D_{\bm{v}} = \sum_{i=1}^{n} v^i \evalat[\bigg]{\pdv{}{x^i}}{p}
\end{equation}
%
The association \(\bm{v} \mapsto D_{\bm{v}}\) offers a way to \newterm{isomorphically} identify tangent vectors with operators on functions.
%
The following makes this rigorous.

\subsection{Derivations}

For each tangent vector \(\bm{v}\) at a point \(p \in \mathbb{R}^n\), the directional derivative at \(p\) gives a map of vector spaces
%
\begin{equation*}
    D_{\bm{v}} \colon C_p^\infty \rightarrow \mathbb{R}
\end{equation*}
%
\(D_{\bm{v}}\) is a linear map that satisfies the \newterm{Leibniz rule}
%
\begin{equation}
    D_{\bm{v}}(fg) = (D_{\bm{v}}f)g(p) + f(p) (D_{\bm{v}}g)
\end{equation}
%
because the partial derivative satisfy the product rule.
%
In general, any linear map \(L\colon C_p^\infty \rightarrow \mathbb{R}\) that satisfies the Leibniz rule is called a \newterm{derivation} at \(p\).
% or a \textit{point derivation} of \(C_p^\infty\). 
%
Denote the set of all derivations at \(p\) by \(\mathcal{D}_p(\mathbb{R}^n)\).
%
\textbf{This set is also a real vector space}.
%

So far we know directional derivatives \(D_{\bm{v}}\) at \(p\) are derivations at \(p\).
%
Thus, there is a map
\begin{align*}
    \phi\colon T_p(\mathbb{R}^n) & \rightarrow \mathcal{D}_p (\mathbb{R}^n) \\
    \bm{v}                       & \mapsto D_{\bm{v}}
\end{align*}
%
\begin{theorem}
    The linear map \(\phi\) is an isomorphism of vector spaces.
\end{theorem}
%
\noindent The implication is that we may identify tangent vectors at \(p\) with derivations at \(p\) (by way of directional derivatives against germs).
%
Under this isomorphism \(T_p(\mathbb{R}^n) \simeq \mathcal{D}_p(\mathbb{R}^n)\), the standard basis \(\left\{ e_1, \dots, e_n \right\}\) for \(T_p(\mathbb{R}^n)\) maps to
%
\begin{equation}
    \left\{\evalat[\bigg]{\pdv{}{x^1}}{p}, \dots, \evalat[\bigg]{\pdv{}{x^n}}{p}  \right\}\label{eqn:tangentbasis}
\end{equation}
%
Therefore from now on we write a tangent vector as
%
\begin{equation}
    \bm{v} = \sum_{i=1}^{n} v^i \evalat[\bigg]{\pdv{}{x^i}}{p}
\end{equation}
%
The point being that, while not as geometrically intuitive as arrows, \(\mathcal{D}_p (\mathbb{R}^n)\) generalizes to manifolds.

\subsection{Vector Fields}

A \newterm{vector field} \(X\) on an open \(U \subset \mathbb{R}^n\) is function that assigns to \(p \in U\) a tangent vector \(X_p \in T_p(\mathbb{R}^n)\).
%
Notice, carefully, that the vector field assigns at each point a vector in the tangent space anchored at that point.
%
Using the tangent basis (eqn.~\eqref{eqn:tangentbasis})
%
\begin{equation}
    X\colon p \mapsto \sum_i a^i(p) \evalat[\bigg]{\pdv{}{x^i}}{p}
\end{equation}
%
Note that both the coefficients \textbf{and} the partial derivatives are evaluated at \(p\).
%
Having said that, we often omit \(p\) in the specification of a vector field when it clear from context.
%
\begin{example}
    On \(\mathbb{R}^n - \{\bm{0}\}\), let \(p = (x,y)\). Then
    \begin{align*}
        X & = \frac{-y}{\sqrt{x^2+y^2}} \pdv{}{x} + \frac{x}{\sqrt{x^2+y^2}} \pdv{}{y} \\
          & = \begin{bmatrix}
            \frac{-y}{\sqrt{x^2+y^2}} \\ \frac{x}{\sqrt{x^2+y^2}}
        \end{bmatrix}                                               \\
          & = \begin{bmatrix}
            \frac{-y}{\sqrt{x^2+y^2}} & \frac{x}{\sqrt{x^2+y^2}}
        \end{bmatrix}^T
    \end{align*}
\end{example}
%
See figure~\ref{fig:vectorfields}
% \loadfig{vecfields}

In general we can identify vector fields with parameterized column vectors
%
\begin{equation}
    X = \sum_i a^i(p) \evalat[\bigg]{\pdv{}{x^i}}{p} 
    \leftrightarrow 
    \begin{bmatrix}
        a^1(p) \\ \vdots \\ a^n(p)
    \end{bmatrix}
\end{equation}

\subsection{Dual Space}
\newcommand{\pdx}[1]{\partial_{x_{#1}}}
\newcommand*{\vwedge}{V^\wedge }
\newcommand*{\vx}{X = \sum a^i \partial_{x_i}}
The \newterm{dual space} \(\vwedge\) of \(V\) is the set of all real-valued linear functions on \(V\) i.e. all \(f \colon V \rightarrow \R\).
%
Elements of \(\vwedge\) are called \newterm{covectors}.

Assume \(V\) is finite dimensional and let \(\{e_1, \dots, e_n\}\) be a basis \(V\).
%
Recall that \(e_i \coloneqq \partial_{x_i}\).
%
Then \(\vx\) for all \(X \in T_p\).
%
Let \(\alpha^i \colon V \rightarrow \mathbb{R}\) be the linear function that picks out the \(i\)th coordinate of a \textbf{vector}, i.e. \(\alpha^i(X) = a^i(p)\).
%
Note that 
\begin{align}
    \alpha^i(\partial_j) &= \alpha^i(1\cdot \partial_j) \\
    &= \begin{cases}
         1 \text{ if } i = j \\
         0 \text{ if } i \neq j
    \end{cases} \\
    &=\delta_j^i
\end{align}
Note that position of indices is important -- upper indices are for covectors.
\begin{proposition}
    \(\{\alpha^i\}\) form a basis for \(\vwedge\).
\end{proposition}
\begin{proof}
    We first prove that \(\{\alpha^i\}\) span \(\vwedge\). If \(f \in V^\wedge\) and \(\vx \in V\), then
    \begin{align}
        f(X) &= \sum a^i f(\pdx{i}) \\
        &= \sum \alpha^i(X) f(\pdx{i}) \\
        &= \sum f(\pdx{i}) \alpha^i (X)
    \end{align}
    which shows that any \(f\) can be expanded as a linear sum of \(\alpha^i\).
    To show linear independence, suppose \(\sum c_i \alpha^i = 0\) with at least one \(c_i\) non-zero. Applying this to an arbitrary \(\pdx{i}\) gives 
    \begin{equation}
        0 = \left(\sum_i c_i \alpha^i   \right)(\pdx{i}) = \sum_i c_i \alpha^i(\pdx{i}) = \sum_i c_i \delta_j^i = c_j
    \end{equation}
    which is a contradiction. Hence \(\alpha^i\) are linearly independent.
\end{proof}
This basis \(\{\alpha^i\}\) for \(\vwedge\) is said to be \textit{dual} to the basis \(\{\pdx{i}\}\) for \(V\).


\begin{example}
   (\newterm{Coordinate functions}) With respect to a basis \(\{\pdx{i}\}\) for \(V\), every \(X \in V\) can be written uniquely as a linear combination \(\vx\) with \(a^i \in \R\). Let \(\{\alpha^i\}\) be the dual basis (i.e. the basis for \(\vwedge\)). Then 
   \begin{align*}
        \alpha^i(X) &= \alpha^i \left( \sum_j a^j \pdx{j} \right) \\
        &= \sum_j a^j \alpha^i(\pdx{j})  \\ 
        &= \sum_j a^j \delta_j^i \\
        &= a^i
   \end{align*}
   Thus, the dual basis \(\{\alpha^i\}\) to \(\{\pdx{i}\}\) is the set of coordinate functions. The sense here is that since tangent vectors are directional derivatives (i.e.\ operators on functions) and the dual space is a mapping from those operators to \(\R\), then a mapping from operators to scalars means hitting an operator with a function (or vice-versa). \textbf{And} the coordinate functions are constant with respect to each other coordinate (and hence partials wrt them are naturally zero).
\end{example}
