%! Suppress = GatherEquations
\anotecontent{vbsuper}{We will often use the verb form "to super resolve" in order to denote the use of one or more such methods.}
\anotecontent{rayleighscriterion}{The amplitude of the diffraction pattern (known as the Airy pattern) of a monochromatic point source through a circular aperture is given by \[I(\theta)=I_{0}\left[{\frac {2J_{1}(ka\sin \theta )}{ka\sin \theta }}\right]^{2}\] where $I_0$ is peak intensity (at the center), $k=\frac{2\pi}{\lambda}$ is the wave number of the light, $\theta$ is the angle of observation, and $J_1$ is the Bessel function of the first kind of order one\cite{goodman2005introduction}. It is maxima/minima of this function that Rayleigh's criterion concerns.}
\anotecontent{shotnoise}{TODO}
\anotecontent{satelliteoptics}{Rayleigh's criterion implies that the angular resolution $R$ of a telescope with optical diameter $D = 2.4\text{m}$ observing visible light (${\sim}500\text{nm}$) is approximately\cite{doi:10.1080.14786447908639684} \[R  \approx 1.220\frac{\lambda}{D} = 1.220 \frac{500\text{nm}}{2.4\text{m}} \approx 0.06 \text{arcsec}\] From an altitude of 250 km this corresponds to a ground sample distance of 6cm. This loss of resolving power is further exacerbated by refraction through turbulent atmosphere\cite{Fried:66}.}
\anotecontent{subpixel}{For example when a point source wholly captured by one sensor element shifts to distributing energy equally amongst the same element and a direct adjacent.}

