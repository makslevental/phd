%! Suppress = GatherEquations
% introduction
\anotecontent{vbsuper}{We will often use the verb form "to super resolve" in order to denote the use of one or more such methods.}
\anotecontent{rayleighscriterion}{The amplitude of the diffraction pattern (known as the Airy pattern) of a monochromatic point source through a circular aperture is given by \[I(\theta)=I_{0}\left[{\frac {2J_{1}(ka\sin \theta )}{ka\sin \theta }}\right]^{2}\] where $I_0$ is peak intensity (at the center), $k=\frac{2\pi}{\lambda}$ is the wave number of the light, $\theta$ is the angle of observation, and $J_1$ is the Bessel function of the first kind of order one\cite{goodman2005introduction}. It is maxima/minima of this function that Rayleigh's criterion concerns.}
\anotecontent{shotnoise}{TODO}
\anotecontent{satelliteoptics}{Rayleigh's criterion implies that the angular resolution $R$ of a telescope with optical diameter $D = 2.4\text{m}$ observing visible light (${\sim}500\text{nm}$) is approximately\cite{doi:10.1080.14786447908639684} \[R  \approx 1.220\frac{\lambda}{D} = 1.220 \frac{500\text{nm}}{2.4\text{m}} \approx 0.06 \text{arcsec}\] From an altitude of 250 km this corresponds to a ground sample distance of 6cm. This loss of resolving power is further exacerbated by refraction through turbulent atmosphere\cite{Fried:66}.}
\anotecontent{subpixel}{For example when a point source wholly captured by one sensor element shifts to distributing energy equally amongst the same element and a direct adjacent.}
% background
\anotecontent{lexico}{Consider an $N$ row $\times$ $M$ column image $x(n, m)$. The lexicographical ordering of the image is the $NM$ length column-wise vectorization of the image: $\bm{X}(Mm + n) = x(n, m)$.}
\anotecontent{patch}{$k \times k$ pixel window, e.g. $3 \times 3$.}
\anotecontent{tensor}{A multidimensional array\cite{Goodfellow:2016:DL:3086952}. Not to be confused with the algebraic object.}
\anotecontent{deep-depletion}{The depletion layer is the region at the Si-SiO$_2$ interface that is characterized by few charge carriers (electrons and holes) that occurs when a gate voltage higher than the flat-band voltage is applied. Deep-depletion is characterized by a depletion region width that is greater than that at equilibrium\cite{semiconductorbook}.}
\anotecontent{inversion}{The term inversion means that the surface is inverted from P type to N type, or electron rich. This process is a slow thermal equilibration\cite{ccdarkcurrent}, is called dark current and at various temperatures/exposure times is a source of noise.}
\anotecontent{depletion}{By applying a positive voltage to the gate holes are repelled and migrate through the body to ground. This leaves a depleted region near the Si-SiO$_2$ interface that acts as an insulator.}
\anotecontent{buriedchannel}{Buried channel as distinct from surface channel. In surface channel MOS capacitors signal charge is stored at the Si-SiO$_2$ interface, which can lead to charge trapping during the charge transfer process.}
\anotecontent{ccd}{TODO}
\anotecontent{pnjunction}{The interface between a p-type semiconductor (excess holes, i.e. positive charge carriers) and an n-type (excess electrons, i.e. negative charge carriers) semiconductor.}
\anotecontent{reversebias}{With the p-type material at a lower voltage than the n-type. This causes both the holes and the electrons to flow away from the junction creating a depletion zone.}
