%
%TODO: work out diffraction circular aperture
%
%TODO: workout poisson noise
%
%TODO: workout conjugate gradients
%
%TODO: workout ransac
%
%TODO: workout belief propagation
%
%TODO: workout auto diff
%
%TODO: workout delauney
%
%TODO: workout splines
%

\section{Appendix}\label{sec:appendix}
\localtableofcontents

\subsubsection{Rayleigh Criterion}

\newcommand*{\E}{\bm{E}}
\newcommand*{\B}{\bm{B}}

Starting with Maxwell's equations in differential form:
\begin{align}
    \nabla \cdot \B = 0 \\
    \nabla \cdot \E = 0 \\
    \nabla \times \B = \frac{1}{c^2} \frac{\partial \E}{\partial t} \\
    \nabla \times \E = - \frac{\partial \B}{\partial t} 
\end{align}
Note that 
\begin{equation}
    \nabla \times \left( \nabla \times \E \right) = \pd{}{t} \left( \nabla \times \B\right) = \frac{1}{c^2} \pdv[2]{\E}{t}
\end{equation}
and with the identity \(\nabla \times \nabla \times \E = \nabla (\nabla\cdot \E) - \Delta \E\) (where \(\Delta \coloneqq \nabla^2\) is the Laplacian) we have the vector E-field \textit{wave equation}
\begin{equation}
    \Delta \E = \frac{1}{c^2} \pdv[2]{\E}{t}
\end{equation}
This decouples each components of the \textit{\textbf{E}} field. We therefore arbitrarily choose the \(z\) component and solve the \textit{scalar wave equation} for \(E \coloneqq E_z\)
\begin{equation}
    \Delta E = \frac{1}{c^2} \pdv[2]{E}{t}\label{eqn:scalarwaveeqn}
\end{equation}
In a medium with refractive index \(n \coloneqq c/v\) (where \(v\) is the speed of light in the medium) eqn.~\eqref{eqn:scalarwaveeqn} becomes
\begin{equation}
    \Delta E = \frac{n^2}{c^2} \pdv[2]{E}{t}\label{eqn:scalarwaveeqnmed}
\end{equation}
We seek \textit{monochromatic} solutions, i.e., 