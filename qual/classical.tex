\subsection{Registration}
\begin{figure}
    \centering
        \forestset{
          direction switch/.style={
            where level>=1{folder, grow'=0}{for children=forked edge},
          },
        }
        \begin{forest}
          direction switch
          [SR
            [MISR
              [Interpolation
                [Adaptive Filtering]
              ]
              [Statistical
                [Back Projection]
                [Markov Random Fields]
                [Bilateral Total Variation]
              ]
            ]
            [SISR
                [Manifold Learning]
                [Compressed Sensing]
                [Projection]
                [Belief Networks]
            ]
          ]
        \end{forest}
    \caption{Taxonomy of classical SR techniques}
    \label{fig:taxonomy}
\end{figure}

Figure~\ref{fig:taxonomy} lays out a rough taxonomy of classical SR algorithms.
%
We cover algorithms from each "genus" and others that don't neatly fit into the taxonomy.
\subsection{Interpolation}
Suppose that $H_k$ is linear spatial\anote{lsi} and time invariant.
%
Suppose further that $A_k$ is affine.
%
Then $H \coloneqq H_k$ commutes with $A_k$\cite{meladcommute} and equation~\ref{eqn:commuteimagingmodel} becomes
\begin{equation}
    \label{eqn:commuteimagingmodel}
    \begin{split}
        X_k &= (D_k \circ A_k \circ H) (Y) + \varepsilon \\
        &= (D_k \circ A_k) (H(Y)) + \varepsilon \\
        &= (D_k \circ A_k) (V) + \varepsilon
    \end{split}
\end{equation}
This naturally suggests interpolation in order to recover $V$ (since $X_k$, in this framing, is simply shifted samples of $V$).
%
Note in this context we use interpolation very broadly, i.e. to connote filling in missing values using neighboring (in some sense --- not necessarily geometrically) values.
%
\begin{figure}
    \centering
    \includegraphics[width=\linewidth]{figures/hrgrid.png}
    \caption{LR image registration on an HR grid\cite{Lin}}
    \label{fig:hrgrid}
\end{figure}
This class of techniques proceed by first registering images on a high resolution grid (see figure~\ref{fig:hrgrid}) then interpolating at the "missing" pixels in the HR grid to recover $V$, and finally denoising and deconvolution (of $H$) to recover $Y$.
%
Since in general consecutive $X_k$ have non-uniform shifts (relative to $X_0$) the interpolation is non-uniform and improvisations on this theme use various weighting schemes for adjacent LR pixels\anote{lrpixel}.

For example Alam et. al\cite{Alam2000} uses weighted nearest neighbors: for every pixel to be interpolated the three nearest pixels are weighted inversely by their distance (according to HR grid distance) and then their weighted sum is assigned to that pixel.
%
This non-uniform interpolation is then followed by application of a Wiener filter whose design is informed by the OTF of the particular imaging system they study (which they do not estimate i.e. they assume they can model accurately).
%
\begin{figure}
    \centering
    \includegraphics[width=.7\linewidth]{figures/delauney.png}
    \caption{Delaunay triangulation for fitting splines at LR pixels\cite{Lertrattanapanich}. $v$ is an LR pixel. Note that $v$ is at $z$ equal to the pixel value}
    \label{fig:delauney}
\end{figure}
Lertrattanapanich et. al\cite{Lertrattanapanich} base their algorithm on interpolants which require knowledge of gradients (e.g. splines) and mediate the non-uniform sampling by using a weighted average (by area) of those gradients in adjacent Delaunay cells; to be precise they produce a Delaunay triangulation of all LR pixels and compute the gradients (see figure~\ref{fig:delauney}) according to
\begin{align*}
    \vec{n} = \sum_{j=1}^k \frac{A_j \vec{n_j}}{A} &\text{ where } A=\sum_{i=1}^k A_i\\
    \frac{\partial z}{\partial x} = -\frac{n_x}{n_z} &\text{ and }  \frac{\partial z}{\partial y} = -\frac{n_y}{n_z}
\end{align*}
Unfortunately this intricate solution is not robust to noise in real images.

A more sophisticated method for non-uniform interpolation uses parametric models for the auto-correlation between LR pixels and the cross-correlation between LR pixels and interpolated pixels to estimate wiener filter weights\cite{wiener}.
%
These weights are then used to average nearby pixel values.
%
The algorithm operates on a sliding \textit{estimation window} whose dimensions $D_x, D_y$ are chosen such that the effective sampling rate exceeds the Nyquist rate for a given $\rho_c$.
\begin{figure}
    \centering
    \includegraphics[width=.7\linewidth]{figures/wiener.png}
    \caption{Wiener filter super resolution estimation window of dimension $D_x \times D_y$ and observation window of dimension $W_x \times W_y$\cite{wiener}}
    \label{fig:wiener}
\end{figure}
The pixel values for the estimation window are a function of the wiener filter weights of nearby LR pixels within an \textit{observation window} whose dimensions $W_x, W_y$ are an integer multiple of $D_x, D_y$ (see figure~\ref{fig:wiener}).
%
The weights $\bm{w}$ are defined as the solution to the minimum mean squared error (MMSE) filter problem, i.e. the finite impulse response (FIR) wiener filter:
\begin{equation}
    \bm{w} = R^{-1}\bm{p}
\end{equation}
where $R$ is the auto-correlation of the LR pixels in the observation window and $\bm{p}$ is the cross-correlation between the pixels to be estimated and the LR pixels.
%
Then $R$ and $\bm{p}$ are both constructed by sampling a parametric model that weights pixels in the observation window according to distance:
%
$R$ is constructed by sampling from
\begin{equation}
    C_1(r) = \sigma_{d}^2 \rho^{r} \ast G(r)
\end{equation}
and $\bm{p}$ is constructed by sampling from
\begin{equation}
    C_2(r) = \sigma_d^2 \rho^{r} \ast G(r) \ast G(-r)
\end{equation}
In the case of $R$, $r$ is distance on the HR grid, $\sigma_d$ is related to the empirical variance of all LR pixels in a given observation window and $G(r)$ is a smoothing kernel (e.g. gaussian).
%
Thus by evaluating $C_1$ for all $r = r(n_1, n_2)$ distances between LR pixels $n_1$, $n_2$ we can construct $R$.
%
Similarly for $\bm{p}$, $r = r(m, n)$ is the distance between pixel-to-be-estimated $m$ and LR pixel $n$.
%
Note that $R$ is an $N \times N$ matrix where $N = K W_x W_y/D_x D_y$, i.e. how many LR pixels there are in the observation window, and $\bm{p}$ is an $N \times 1$ column vector uniquely computed for each pixel in the estimation window.
%
The scheme is effective but suffers from issues with the spatial isotropy of the auto-correlation and cross-correlation models.

One of the most sophisticated of these non-uniform interpolation schemes employs the kernel regression framework and \textit{steering kernels}\cite{Takeda2007}.
%
In this context we start with all $X_k$ registered to a common HR grid and consider pixel values $Y_i = Y(\bm{x}_i)$ at pixel coordinates $\bm{x}_i = (x_{i1},x_{i2})$ as the measured data pairs $(\bm{x}_i, Y_i)$.
%
Recall that kernel regression frames the estimation problem as
\newcommand*{\bx}{\bm{x}}
\newcommand*{\bxi}{\bm{x}_i}
\newcommand*{\delx}{\bx - \bxi}
\newcommand*{\zbx}{Z(\bx)}
\newcommand*{\zbxi}{Z(\bxi)}
\newcommand*{\bb}{\bm{\beta}}
\newcommand*{\hzbx}{\hat{Z}(\bx)}
\begin{equation}
    Y_i = Z(\bx_i) + \varepsilon
\end{equation}
where $Z$ is the to-be-estimated \textit{regression function} that "predicts" $Y$ as a function of $\bx$.
Then the Nadaraya–Watson estimator (NWE)\cite{Nadaraya} $\hat{Z}$ for $Z$ is
\begin{equation}
    \hat{Z}(\bx) = \frac{\sum_{i=1}^{P}K(\delx)Y_i}{\sum_{i=1}^{P}K(\delx)}
\end{equation}
where $P$ indexes over all pixels in the HR grid and $K$ is a \textit{kernel function} whose purpose is to decay the contribution of $\bxi$ if it's in some sense far from $\bx$.
%
\begin{figure}
    \centering
    \includegraphics[width=.8\linewidth]{figures/footprint.png}
    \caption{Kernel footprint as a function of sample density\cite{Takeda2007}}
    \label{fig:footprint}
\end{figure}
\begin{figure}
    \centering
    \includegraphics[width=\linewidth]{figures/steering.png}
    \caption{Adapting kernel shape as a function of local directed structure\cite{Takeda2007}}
    \label{fig:steering}
\end{figure}
In conventional kernel regression $K$ might be any non-negative, symmetric, unimodal\cite{wand1994kernel} function with augmented with an additional $h$ parameter that controls the "bandwidth" or "footprint" of the kernel, i.e.
\begin{equation}
    K_h(\delx) = \frac{1}{h}K\left( h^{-1}(\delx) \right)
\end{equation}
This bandwidth parameter $h$ can be generalized to a \textit{smoothing kernel} $H$ in order to make $K = K_H$ adaptive to the local structure of the pixels, e.g. to have larger footprints in sparsely sampled regions and have smaller footprints in densely sampled regions (see figure~\ref{fig:footprint}).
%
Ultimately though it is desirable to have kernels that can adapt to directed structure in the image, i.e. "steerable" kernels that filter strongly along an edge and weakly across an edge.
%
This is accomplished by, for example, using a Gaussian as the kernel:
\begin{equation}
    K_{H_i}(\delx) \propto \frac{\exp\left\{ -(\delx)^T H^{-1}_i (\delx) \right\}}{\sqrt{\det{H_i}}}
\end{equation}
and identifying $H_i$ with $\nabla^2 \zbxi$ (since gradients capture edge structure).
%
An estimate $\hat{H}_i$ of $\nabla^2 \zbxi$ can be obtained by looking at covariances of empirical gradients (i.e. the HR grid registered image convolved with a difference filter).
%
Unfortunately this is a naive estimate that is often rank deficient or unstable (both leading to instances where $\hat{H}_i$ isn't invertible).
%
One solution is to parameterize $H_i$:
\[
    H_i = \gamma_i U_{\theta_i} \Lambda_{\sigma_i} U_{\theta_i}^T
\]
where $U_{\theta_i}$ is a rotation matrix, $\Lambda_{\sigma_i} = \text{diag}\left( \sigma_i, \sigma_i^{-1} \right)$ is an "elongation" matrix, and $\gamma_i$ is a scaling parameter, with each of $\gamma_i, \theta_i, \sigma_i$ estimated from the data in a more robust way.
%
Another solution is to use a multi-scale Kalman filter to estimate $H_i$\cite{XiaoGuang}.

In general non-uniform interpolation techniques are intuitive and typically (relatively) computationally efficient but they assume an unrealistic observation model (namely that affine flow).
\subsection{Estimation}
%
%For example a full Bayesian treatment would construct an estimate $\hat{Y}$ of $Y$ as
%\begin{align}
%    \hat{Y} &= \underset{Z}{\text{argmax}}~P\left( Y | X_0, X_1, \mathellipsis, X_k \right) \nonumber \\
%    &= \underset{Y}{\text{argmax}} \int_{H, A} P\left( Y, H, A | \bm{X}\right) \nonumber \\
%    &= \underset{Y}{\text{argmax}} \int_{H, A} \frac{P\left(\bm{X} | Y, H, A \right)P(Y) P(H, A)}{P(\bm{X})}\label{eqn:indepym} \\
%    &= \underset{Y}{\text{argmax}} \int_{H, A} P\left(\bm{X} | Y, H, A \right)P(Y) P(H, A) \label{eqn:maxx}
%\end{align}
%where in eqn.~\ref{eqn:indepym} we've used the independence of $Y$ and $H,A$\cite{Hardie1997} and in eqn.~\ref{eqn:maxx} we've used that $\bm{X}$ is a constant with respect to the maximization.
%%
%While there exist reasonable priors for $Y$, marginalizing over $H, A$ is still difficult due to the high-dimensionality of each.
%%
%Therefore assuming $H, A$ can be estimated independently as $\hat{H}, \hat{A}$, eqn.~\ref{eqn:maxx} becomes
%\begin{equation}
%    \hat{Y} = \underset{Y}{\text{argmax}}~P\left(\bm{X} | Y; \hat{H}, \hat{A} \right) P(Y)
%    \label{eqn:map}
%\end{equation}
%which casts $\hat{Y}$ the standard maximum a posteriori (MAP) of $Y$.
%%
%If we further assume a uniform prior over $Y$ eqn.~\ref{eqn:map} reduces to
%\begin{equation}
%    \hat{Y} = \underset{Y}{\text{argmax}}~P\left(\bm{X} | Y; \hat{H}, \hat{A}\right)
%    \label{eqn:mle}
%\end{equation}
%i.e. $\hat{Y}$ is the maximum likelihood estimate of $Y$.
%%
%Note that an equivalent formulation of both MLE and MAP minimize the negative log-likelihood instead of maximizing the likelihood:
%\begin{align}
%    \hat{Y} &= \underset{Y}{\text{argmax}}~P\left(\bm{X} | Y; \hat{H}, \hat{A}\right) P(Y) \nonumber \\
%    &=  \underset{Y}{\text{argmin}}\left[ -\log{P\left(\bm{X} | Y; \hat{H}, \hat{A} \right)} -\log{P(Y)} \right]
%    \label{eqn:logmap}
%\end{align}
%%
%In either case (MLE or MAP) various choices for the conditional distribution $P\left(\bm{X} | Y; \hat{H}, \hat{A}\right)$ and the prior distribution $P(Y)$ (and consequent choice of maximization strategies) characterize this class of SR techniques.
%%
%Typically $\varepsilon$ in eqn.~\ref{eqn:imagingmodel} is distributed $N(0, \sigma)$, i.e. zero-mean unit variance Gaussian, which constrains the conditional distribution to be Gaussian.
%%
%Therefore eqn.~\ref{eqn:mle} becomes
%\begin{equation}
%    \hat{Y} = \underset{Y}{\text{argmin}}\left\| X-\hat{H}\hat{A}Y \right\|^2
%    \label{eqn:minmle}
%\end{equation}
%and after a suitable choice for $P(Y)$, (e.g. Gibbs\cite{Hardie1997}):
%\[
%    P(Y) = \frac{1}{Z}e^{-\alpha B(Y)}
%\]
%eqn.~\ref{eqn:logmap} becomes
%\begin{equation}
%    \hat{Y} = \underset{Y}{\text{argmin}}\left[ \left\| X-\hat{H}\hat{A}Y \right\|^2 +\lambda B(Y)\right]
%    \label{eqn:minmap}
%\end{equation}
%where $\lambda$ absorbs $\alpha$ from $P(Y)$ and $\sigma$ from $\varepsilon$.
Statistical estimation methods cast SR as an inference problem.
%
One of the earliest successful SR algorithm\cite{Irani1991ImprovingRB} proposed an iterative algorithm inspired by the back-projection method commonly used to reconstruct 2-D objects from 1-D projections in computer-aided tomography.
%
Recall that we assumed that motion and blur commute (see eqn.~\ref{eqn:commuteimagingmodel}).
%
Then the idea is to take the current estimate of the HR image $\hat{V}^{i}$, see if after motion and down-sampling it is near the LR samples $X_k$, and add a correction when it is not:
\begin{equation}
    \hat{V}^{i+1} = \hat{V}^i + \sum_k (D_k \circ A_k)^{-1}\left( (D_k \circ A_k)(\hat{V}^i) - X_k \right)
\end{equation}
where $\hat{V}^i$ is the current estimate of the blurred HR image, $(D_k \circ A_k)(\hat{V}^i)$ is the projection of the current estimate to low resolution, and $(D_k \circ A_k)^{-1}\left( (D_k \circ A_k)(\hat{V}^i) - X_k \right)$ is the \textit{back-projection}.
%
Irani et al.\cite{Irani1991ImprovingRB} also convolve the back-projection with a smoothing kernel as a form of regularization since the estimation problem is in general ill-posed (there are many $\hat{V}^{i}$ that will project down to a pixel-distance neighbor of $X_k$)
%
It can be shown\cite{Elad1996} that $\hat{V}$ is none other than the maximum likelihood estimate for $V$.


%
%\begin{equation}
%    \hat{Y} = \underset{Y}{\text{argmin}}\left[ \left\| X-HAY \right\|^2 + \lambda B(Y) \right]
%    \label{eqn:minmap}
%\end{equation}

\subsection{Example based}
