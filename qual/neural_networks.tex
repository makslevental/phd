%\input{figures/net}
\section{Artificial Neural Networks}\label{sec:neural-networks}
\localtableofcontents
Artificial Neural Network algorithms for SR are all based on a Convolutional Neural Network (CNN) architecture.
%
We first briefly review ANNs in general and CNNs in particular and then explore the recent advances in Deep Learning (DL) for SR.
%
\subsection{Basics}
\begin{figure*}[!htbp]
    % MLP
    \centering
    \tikzstyle{inputNode}=[draw,fill=sail,circle,minimum size=10pt,inner sep=2pt]
    \tikzstyle{hiddenNode}=[draw,fill=snowymint,circle,minimum size=10pt,inner sep=2pt]
    \tikzstyle{outputNode}=[draw,fill=plum,circle,minimum size=10pt,inner sep=2pt]
    \tikzstyle{stateTransition}=[-stealth, thick]
    \begin{subfigure}[b]{0.49\textwidth}
        \centering
        \begin{tikzpicture}
            \node[draw,fill=plum, circle,minimum size=25pt,inner sep=0pt] (x) at (0,0) {$\Sigma$ $\sigma$};
            \draw[dashed] (0,-0.43) -- (0,0.43);

            \node[inputNode] (x0) at (-2, 1.5) {$ x_1$};
            \node[inputNode] (x1) at (-2, 0.75) {$ x_2$};
            \node[inputNode] (x2) at (-2, 0) {$ x_3$};
            \node[inputNode] (x3) at (-2, -0.75) {$ x_4$};
            \node[inputNode] (xn) at (-2, -1.95) {$ x_n$};

            \draw[stateTransition] (x0) to[out=0,in=120] node [midway, sloped, above] {$w_1$} (x);
            \draw[stateTransition] (x1) to[out=0,in=150] node [midway, sloped, above] {$w_2$} (x);
            \draw[stateTransition] (x2) to[out=0,in=180] node [midway, sloped, above] {$w_3$} (x);
            \draw[stateTransition] (x3) to[out=0,in=210] node [midway, sloped, above] {$w_4$} (x);
            \draw[stateTransition] (xn) to[out=0,in=240] node [midway, sloped, above] {$w_n$} (x);
            \draw[stateTransition] (x) -- (4,0) node [midway,above] {$z(\mathbf{x}) = \sigma\left(\sum\limits_{i=1}^{n}{w_ix_i} +  b\right)$};
            \node (dots) at (-2, -1.25) {$\vdots$};
        \end{tikzpicture}
        \caption{Single Artificial Neuron function.}\label{fig:singleann}
    \end{subfigure}
    \begin{subfigure}[b]{0.49\textwidth}
        \centering
        \begin{tikzpicture}

            \node[inputNode, thick] (i1) at (6, 0.75) {$x_1$};
            \node[inputNode, thick] (i2) at (6, 0) {$x_2$};
            \node[] (i4) at (6, -0.75) {$\LARGE \vdots$};
            \node[inputNode, thick] (i3) at (6, -1.5) {$x_n$};

            \node[hiddenNode, thick] (h1) at (8, 1.5) {$z_1$};
            \node[hiddenNode, thick] (h2) at (8, 0.75) {$z_2$};
            \node[hiddenNode, thick] (h3) at (8, 0) {$z_3$};
            \node[] (h4) at (8, -0.75) {$\LARGE \vdots$};
            \node[]  at (8, -1.5) {$\LARGE \vdots$};
            \node[hiddenNode, thick] (h5) at (8, -2.25) {$z_m$};

            \node[outputNode, thick] (o2) at (10, 0) {$\Sigma$ $\sigma$};
            \draw[dashed] (10,-0.43) -- (10,0.43);

            \draw[stateTransition] (i1) -- (h1);
            \draw[stateTransition] (i1) -- (h2);
            \draw[stateTransition] (i1) -- (h3);
            \draw[stateTransition] (i1) -- (h5);
            \draw[stateTransition] (i2) -- (h1);
            \draw[stateTransition] (i2) -- (h2);
            \draw[stateTransition] (i2) -- (h3);
            \draw[stateTransition] (i2) -- (h5);
            \draw[stateTransition] (i3) -- (h1);
            \draw[stateTransition] (i3) -- (h2);
            \draw[stateTransition] (i3) -- (h3);
            \draw[stateTransition] (i3) -- (h5);

            \draw[stateTransition] (h1) -- (o2);
            \draw[stateTransition] (h2) -- (o2);
            \draw[stateTransition] (h3) -- (o2);
            \draw[stateTransition] (h5) -- (o2);

            \node[above=of i1, align=center] (l1) {\footnotesize $n$ Neuron \\ \footnotesize Input \\ \footnotesize layer};
            \node[right=1.3em of l1, align=center] (l2) {\footnotesize $m$ Neuron \\ \footnotesize Hidden \\ \footnotesize layer};
            \node[right=2.3em of l2, align=center] (l3) {\footnotesize Output \\ \footnotesize layer};

            \draw[stateTransition] (o2) -- node[midway,above] {$y(\mathbf{x}) = \sigma'\left(\sum\limits_{j=1}^{m}{w'_jz_j} + b'\right)$} (13, 0);
        \end{tikzpicture}
        \caption{Multi-layer ANN.}\label{fig:multiann}
    \end{subfigure}
    \caption{Artificial Neural Network representation.}\label{fig:ann}
\end{figure*}

An Artificial Neural Network (ANN or NN) is a function specified by compositions of elementary functions called \textit{artificial neurons}\anote{ann} (or simply neurons).
%
Neurons consist of a set of inputs \(\mathbf{x} = (x_1, x_2, \dots, x_n)\), an aggregation (typically linear combination), and an activation function \(\sigma\), which acts as a thresholding mechanism.
%
For example, the simplest function that qualifies as a neuron is a linear function:
\begin{equation}
    z(\mathbf{x}) = w_1 x_1 + w_2 x_2 = \sum_i w_i x_i
    \label{eqn:simpleann}
\end{equation}
where the the activation function is the trivial one i.e., the identity.
%
Minsky \etal\cite{minsky2017perceptrons} famously proved that neurons that don't include a non-linear activation function have very little representational power (i.e., they're unable to represent functions as simple as even \(\operatorname{XOR}\)\anote{xor}.) and those that do are universal representers (i.e., given enough layers and neurons they're able to represent functions of arbitrary complexity).
%
Common non-trivial, i.e., non-linear, activation functions are the sigmoid function
\begin{equation}
    S(x)={\frac {1}{1+e^{-x}}}={\frac {e^{x}}{e^{x}+1}}
\end{equation}
or the hyperbolic tangent function
\begin{equation}
    \tanh x={\frac {\sinh x}{\cosh x}}={\frac {e^{x}-e^{-x}}{e^{x}+e^{-x}}}={\frac {e^{2x}-1}{e^{2x}+1}}
\end{equation}
or the piece-wise defined \textit{rectified linear unit} (\(\operatorname{ReLU}\))
\begin{align}
    \operatorname{ReLU}(x) & \coloneqq \begin{cases}x&{\text{if }}x>0,\\0&{\text{otherwise}}\end{cases} \\
                           & = \max(0, x)
\end{align}
%
Note that eqn.~\eqref{eqn:simpleann} passes through the origin \((0,0,0)\) since it has no constant term; in the parlance of machine learning the neuron is missing a bias term\anote{bias} \(b\):
\begin{equation}
    z(\mathbf{x}) = \sum_i w_i x_i + b
    \label{eqn:linearregr}
\end{equation}
%
Neurons can be represented as directed graphs where a vertex represent an input or a neuron and edges represent the weights in the linear combination (see figure~\ref{fig:singleann}).
%
ANNs are then assemblies of neurons grouped into \textit{layers} with the layers composed by applying neurons to outputs from immediately preceding layers.
%
Those layers that are not input or output layers are denoted \textit{hidden} layers.
%
For example the ANN specified in figure~\ref{fig:multiann} represents the function
\begin{equation}
    \begin{split}
        y(\mathbf{x}) &= \sigma' \left( \sum_j w'_j z_j(\mathbf{x}) + b' \right) \\
        &=  \sigma' \left( \sum_{j=1}^m w'_j \sigma\left(\sum_{i=1}^n w_i x_i + b_j\right) + b' \right)
    \end{split}
\end{equation}

If ANNs were simply another way to diagrammatically represent non-linear functions they would be fairly uninteresting.
%
In fact ANNs are comprised by their definition and a \textit{learning} method.
%
The learning method enables the function to approximate some other function, by adjusting the weights \(w_i\), given \textit{training} pairs \(\left\{ \mathbf{x}_k, t_k \right\}\) where \(\mathbf{x}_k\) is the \(k\)th training \textit{sample} and \(t_k\) is the \(k\)th training \textit{target}.
%
The most common such learning rule is called the Delta rule\cite{widrow1960adaptive} for a single neuron, which can be derived from minimizing the the squared error \textit{loss} with respect to each of the weights for a given training pair \((\mathbf{x}_k, t_k)\):
\begin{equation}
    L(w_1, \dots, w_n) = \sum_k \frac{1}{2} (t_k - y(\mathbf{x}_k))^2
\end{equation}
and hence
\begin{equation}
    \pd{L}{w_i} = - \sum_k(t_k-y(\mathbf{x}_k))\cdot y'\cdot x_{ik}
\end{equation}
where here by \(y'\) we mean the derivative of the activation function with respect to its argument and by \(x_{ik}\) we mean the \(i\)th input \(x_i\) of the \(k\)th training sample.
%
Hence, by gradient descent the weights \(w_i\) should be adjusted in the opposite direction of \(\pd{L}{w_i}\) and so we have the weight adjustment rule
\begin{equation}
    \Delta w_i = \alpha \cdot \sum_k(t_k-y(\mathbf{x}_k))\cdot \sigma'\cdot x_{ik}
    \label{eqn:batchupdate}
\end{equation}
where \(\alpha\) is a small constant called the \textit{learning rate}.

The Delta rule is essentially the chain rule as applied to ANNs. 
%
In general computing the partial derivatives \(\pd{L}{w_i}\) for a deep (many layers) and wide (many neurons in each layer) network is onerous.
%
\begin{figure}[!htbp]
    \centering
    \begin{subfigure}[b]{.49\textwidth}
        \centering
        \begin{tikzpicture}[]
            \def\pindist{35pt}
            \def\nodesize{38pt}

            \tikzstyle{every pin edge}=[signal]
            \tikzstyle{annot} = [text width=4em, text centered]

            \node[hiddennode, text width=\nodesize, minimum size=\nodesize,
                pin={[pin edge={latex-}, pin distance=\pindist]above left:$w_1 x_1$},
                pin={[pin edge={latex-}, pin distance=\pindist]below left:$w_2 x_2$},
                pin={[pin edge={-latex}, pin distance=\pindist]right:$y$}
            ] (N1) at (-100pt,0) {$\Sigma\quad \sigma$};
            \draw[dashed] (-100pt,-0.55) -- (-100pt,0.55);
        \end{tikzpicture}
        \caption{Forward Pass.}
    \end{subfigure}
    \vskip\baselineskip
    \begin{subfigure}[b]{.49\textwidth}
        \centering
        \begin{tikzpicture}[]
            \def\pindist{35pt}
            \def\nodesize{38pt}
            \tikzstyle{every pin edge}=[signal]
            \tikzstyle{annot} = [text width=4em, text centered]

            \node[hiddennode, text width=\nodesize, minimum size=\nodesize,
                pin={[pin edge={-latex}, pin distance=\pindist]above left:$\frac{\partial L}{\partial w_1}=\frac{\partial L}{\partial y}\frac{\partial y}{\partial w_1}$},
                pin={[pin edge={-latex}, pin distance=\pindist]below left:$\frac{\partial L}{\partial w_2}=\frac{\partial L}{\partial y}\frac{\partial y}{\partial w_2}$},
                pin={[pin edge={latex-}, pin distance=\pindist]right:$\frac{\partial L}{\partial y}$}
            ] (N2) at (+120pt,0) {$\dif \sigma$};
        \end{tikzpicture}
        \caption{Backwards Pass. Note that \(\pd{L}{y}\) can be reused when computing both \(\pd{L}{w_1}\) and \(\pd{L}{w_2}\).}
    \end{subfigure}
    \caption{Back-propagation computation of derivatives.}\label{fig:backprop}
\end{figure}

To mititage the effect of this combinatorial explosion of dependencies between the weights Rumelhart \etal\cite{rumelhart1988learning} popularlized a technique called \textit{back-propagation}\anote{backprop} or simply backprop (see figure~\ref{fig:backprop}).
%
Another inefficiency of the Delta rule is that it requires evaluating the ANN on the entire batch of samples in order to compute the adjustment \(\Delta w_i\).
%
For large training sets (on the order of millions of samples) this is infeasible due to memory limitations.
%
Stochastic Gradient Descent (SGD) replaces computing the \textit{batch loss} \(\sum_k(t_k-y(\mathbf{x}_k))\) in eqn.~\eqref{eqn:batchupdate} with an iterative update to \(w_i\):
\begin{equation}
    w_i^k = w_i^{k-1} + \alpha \cdot (t_k-y(\mathbf{x}_k))\cdot \sigma'\cdot x_{ik}
    \label{eqn:sgd}
\end{equation}
Equation~\eqref{eqn:sgd} is evaluated for each of the \(k\) samples sequentially and therefore saves having to store all training samples in memory.
\subsection{Convolutional Neural Networks}

\subsection{Deep Neural Networks}
\input{figures/neural_networks/deep_archs}