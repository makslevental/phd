\begin{figure*}
    \centering
    \begin{tikzpicture}
        \tikzset{%
            aperture/.pic={
                    \path [draw=black, rotate=#1-90, pattern=north east lines]
                    (-3/4,-3/2) rectangle (3/4,3/2);
                    \path [draw=black, rotate=#1-90, scale=1/8]
                    (0,14) -- ++(-1,0) -- ++(0,2) -- ++(-1,0) -- ++(2,2)
                    -- ++(2,-2) -- ++(-1,0) -- ++(0,-2) -- cycle;
                    \path [fill=white, draw=black, even odd rule]
                    circle [radius=2/3] (-1,-1) rectangle (1,1);
                }
        }
        \path (0,3) node {} (0,0) pic {aperture={135}};
        \path (4,3) node {} (4,0) pic {aperture={90}};
        \path (8,3) node {} (8,0) pic {aperture={180}};
    \end{tikzpicture}
    \caption[]{Barber-poll illusion. Each of the motions appear to be identitical for approriately chosen velocities.}\label{fig:barberpoll}
\end{figure*}