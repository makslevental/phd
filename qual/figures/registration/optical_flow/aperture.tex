\begin{figure*}[!htbp]
    \tikzset{
        aperture/.pic={
                \pgfmathsetmacro\cs{sqrt(1.5*cos(#1-45)^2/2)};
                \path [draw=black, rotate=#1, pattern=north east lines]
                    (-6/4,-6/2) rectangle (6/4,6/2);
                \path [fill=white, draw=black, even odd rule]
                    circle [radius=4/3] (-2,-2) rectangle (2,2);

                \draw[bluearrow, rotate=#1, line width=0.5mm] (0,0) -- node [above] {\small{\(\mathbf{u}\)}} (0,1.5);
                \draw[redarrow, line width=0.5mm] (0,0) -- node [above] {\small{\(\nabla X\)}} (-1.5,1.5);
                \draw[greenarrow, line width=0.5mm] (0,0) -- node [left] {\small{\(\nabla X \cdot \mathbf{u}\)}} (-\cs,\cs);
            }
    }
    \centering
    \begin{subfigure}[b]{0.3\textwidth}
        \centering
        \begin{adjustbox}{height=\textwidth}
            \begin{tikzpicture}
                \pic {aperture={45}};
            \end{tikzpicture}
        \end{adjustbox}
        \caption{Diagonal motion. \(\nabla X \cdot \mathbf{u}\) is maximum and therefore optical flow directly corresponds to motion.}
        \label{fig:diagbarber}
    \end{subfigure}
    \hfill
    \begin{subfigure}[b]{0.3\textwidth}
        \centering
        \begin{adjustbox}{height=\textwidth}
            \begin{tikzpicture}
                \pic {aperture={0}};
            \end{tikzpicture}
        \end{adjustbox}
        \caption{Vertical motion. \(\nabla X \cdot \mathbf{u}\) is less than maximum and therefore optical flow fails to capture all of the motion.}
        \label{fig:vertbarber}
    \end{subfigure}
    \hfill
    \begin{subfigure}[b]{0.3\textwidth}
        \centering
        \raisebox{22pt}{
            \begin{adjustbox}{width=\textwidth}
                \begin{tikzpicture}
                    \pic {aperture={90}};
                \end{tikzpicture}
            \end{adjustbox}
        }
        \caption{Horizontal motion. \(\nabla X \cdot \mathbf{u}\) is less than maximum and therefore optical flow fails to capture all of the motion.}
        \label{fig:horizbarber}
    \end{subfigure}
    \caption{Barber-poll illusion. Each of the three motion measure as vertical optical flow.}
    \label{fig:barberpoll}
\end{figure*}