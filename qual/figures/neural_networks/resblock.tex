\begin{figure}
    \centering
    \begin{subfigure}[b]{.49\textwidth}
        \centering
        \begin{tikzpicture}[start chain=going below, node distance=12pt,
            point/.append style={on chain, join=by {signal}},
            layer/.append style={on chain, join=by {signal}},
            branch/.append style={on chain, join=by {signal, -}},
            ]
            \def\branchy{20pt}

            \node[activation] {Layer 1};
            \node[branch] (input) {};
            \node[bn, xshift=\layerwidth/2+16pt, yshift=-\branchy] {Layer 2};
            \node[layer, xshift=-\layerwidth/2-16pt, yshift=-\branchy] (add) {Layer 3};
            \node[point] {Output};
            \draw[signal] (input) -- (add);
        \end{tikzpicture}
    \caption{Skip connection connecting non-adjacent layers.}\label{fig:skipconnection}
    \end{subfigure}
    \vskip\baselineskip
    \begin{subfigure}[b]{.49\textwidth}
        \centering
        \begin{tikzpicture}[start chain=going below, node distance=12pt,
            point/.append style={on chain, join=by {signal}},
            layer/.append style={on chain, join=by {signal}},
            branch/.append style={on chain, join=by {signal, -}},
        ]
        \def\branchy{30pt}
        
        \node[point] {Input};
        \node[branch] (input) {};
        \node[conv, xshift=\layerwidth/2+16pt, yshift=-\branchy] {Conv};
        \node[bn] {BatchNorm};
        \node[activation] {ReLU};
        \node[conv] {Conv};
        \node[bn] {BatchNorm};
        \node[layer, xshift=-\layerwidth/2-16pt, yshift=-\branchy] (add) {Addition};
        \node[point] {Output};
        \draw[signal] (input) -- (add);
    \end{tikzpicture}
    \caption{A more sophisticated use of skip connections called a ResBlock.}\label{fig:resblock}
    \end{subfigure}

\end{figure}