\documentclass[journal]{IEEEtran}
\usepackage[caption=false]{subfig}
\usepackage{amssymb}
\usepackage{cite}
\usepackage{textcomp}
\usepackage[pdftex]{graphicx}
%\graphicspath{{images/submission-GeoScience/}}
\usepackage{amsmath}
% Note that the amsmath package sets \interdisplaylinepenalty to 10000
% thus preventing page breaks from occurring within multiline equations. Use:
\interdisplaylinepenalty=2500
%\hyphenation{op-tical net-works semi-conduc-tor}

\begin{document}
    \title{Super Resolution for Automated Target Recognition}
    \author{Maksim Levental}
    \maketitle

    \begin{abstract}
        Super resolution (SR) is the process of producing high-resolution (HR) images from low-resolution (LR)
        images while preserving ground truth about the subject matter of the images and potentially
        inferring more such truth. Algorithms that successfully carry out
        such a process are broadly useful in all circumstances where HR imagery is either difficult or impossible
        to obtain. In particular we look towards super resolving images collected using longwave infrared (LWIR)
        cameras; high resolution sensors for such cameras do not currently exist. We present an exposition of
        motivations and concepts of super resolution in general and current techniques, with a qualitative comparison of
        such techniques. Finally we suggest directions for future research, in particular with applications to LWIR images.
    \end{abstract}

    \section{Introduction}
    \section{Background}
    \section{Classical Algorithms}
    \section{Deep Learning Algorithms}
    \section{Future Research}
    \section{Conclusion}

    \section*{Acknowledgment}

%    I would like to thank my wife, Cathy, for her support. I would also like to thank the members of my committee, including my advisor, Dr. Joseph Wilson, along with Dr. Benjamin Lok, Dr. Paul Gader and Dr. Chelsey Simmons for investing their time into this work.


    \bibliographystyle{IEEEtran}
    \bibliography{quals}

\end{document}



