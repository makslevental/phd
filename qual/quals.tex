\documentclass[journal]{IEEEtran}
\usepackage{catchfilebetweentags}
\newcommand{\loadeqn}[1]{%
    \ExecuteMetaData[equations.tex]{eq#1}%
}
\newcommand{\loadfig}[1]{%
    \ExecuteMetaData[figures.tex]{fig#1}%
}
\usepackage[margin=0.7in]{geometry}
\usepackage[parfill]{parskip}
\usepackage[backend=biber, bibencoding=utf8, style=ieee, backref=true]{biblatex}
\usepackage[hyperfootnotes=false,hyperindex=true]{hyperref}
\usepackage[utf8]{inputenc}
\usepackage{amsmath,amssymb,amsfonts,amsthm} % math
\usepackage{bm} % \bm 
\usepackage{physics} % \pdv
\usepackage{mathtools} % coloneqq
\usepackage{imakeidx} % index
\usepackage{etoc} % localtabelofcontents
\usepackage{tikz}
\usetikzlibrary{}

\usepackage{pgfplots}
\usepackage{subcaption}
\DeclareCaptionFormat{subfig}{#1#2#3}
\DeclareCaptionSubType{figure}
\captionsetup[subfigure]{format=subfig,labelsep=space,labelformat=parens}


\newcommand{\newterm}[1]{\textit{#1}\index{#1}}
\newcommand{\Xk}{X_k}
\newcommand{\Xkone}{X_{k+1}}
\newcommand{\bx}{\bm{x}}
\newcommand{\bX}{\bm{X}}
\newcommand{\bxi}{\bm{x}_i}
\newcommand{\delx}{\bx - \bxi}
\newcommand{\by}{\bm{y}}
\newcommand{\bY}{\bm{Y}}
\newcommand{\byi}{\bm{y}_i}
\newcommand{\dely}{\by - \byi}
\newcommand{\zbx}{Z(\bx)}
\newcommand{\zbxi}{Z(\bxi)}
\newcommand{\bb}{\bm{\beta}}
\newcommand{\hzbx}{\hat{Z}(\bx)}
\NewDocumentCommand{\evalat}{sO{\big}mm}{%
  \IfBooleanTF{#1}
   {\mleft. #3 \mright|_{#4}}
   {#3#2|_{#4}}%
}
\theoremstyle{definition}
\newtheorem{theorem}{Theorem}[section]
\theoremstyle{definition}
\newtheorem{example}{Example}[section]


\makeindex
%! Suppress = GatherEquations
% introduction
\anotecontent{hom}{
If \(A\) and \(B\) are modules over a ring \(R\), then \(\operatorname{Hom}_R(A,B)\) is the set of all \(R\)-module homomorphisms \(f\colon A \rightarrow B\). In this case \(V\) is a module over \(\R\) and \(\R\) is trivially a module over \(\R\). 
}
\anotecontent{alternating}{An alternating function is one that changes signs if arguments are transposed (e.g. cross-product or determinant).}

\begin{document}
\title{Super Resolution for Automated Target Recognition}
\author{Maksim Levental}
\maketitle

\begin{abstract}
	Super resolution is the process of producing high-resolution images from low-resolution images while preserving ground truth about the subject matter of the images and potentially inferring more such truth.
	%
	Algorithms that successfully carry out such a process are broadly useful in all circumstances where high-resolution imagery is either difficult or impossible to obtain.
	%
	In particular we look towards super resolving images collected using longwave infrared cameras since high resolution sensors for such cameras do not currently exist.
	%
	We present an exposition of motivations and concepts of super resolution in general and current techniques, with a qualitative comparison of such techniques.
	%
	Finally we suggest directions for future research.
\end{abstract}
\tableofcontents
\section{Introduction}\label{sec:introduction}
\localtableofcontents
Super-resolution (SR) is a collection of methods\anote{vbsuper} that augment the resolving power of an imaging system.
%
Here, and in the forthcoming, by resolving power we mean the ability of an imaging device to distinguish distinct but proximal objects in the scene.
%
If such objects are modeled as point sources of light then the resolving power of the imaging system is defined by Rayleigh's criterion: two point sources are considered \textit{resolved} when the first diffraction maximum\anote{rayleighscriterion} of one point source (at most) coincides with the first minimum of the other (see figure~\ref{fig:rayleigh}).
\begin{figure}
    \center
    \includegraphics[width=.5\linewidth]{figures/rayleigh.png}
    \caption{Rayleigh's criterion\cite{rayleigh}}
    \label{fig:rayleigh}
\end{figure}

SR techniques yield high-resolution (HR) images from one or more observed low-resolution (LR) images by restoring lost fine details and reversing degradations produced by imperfect imaging systems.
%
In the case when a single LR source image is used to construct the HR correspondent, the techniques are referred to as single-image-super-resolution (SISR) techniques.
%
These techniques typically operate by either learning some mapping from low resolution chips (uniform partitions of the image, e.g. $3\times 3$ pixels) to higher resolution chips that are highly similar (according to some metrics) and which obey regularity constraints (e.g. agreement at edges).
%
In the case when multiple LR source images are used to construct the single HR correspondent, the techniques are referred to as multiple-image-super-resolution (MISR) techniques.
%
MISR techniques rely on non-redundant and yet pertinent information in multiple images of the same scene (see figure~\ref{fig:misr}).
\begin{figure}
    \includegraphics[width=\linewidth,keepaspectratio]{figures/misr.png}
    \caption{Multiple image super resolution\cite{misr}}
    \label{fig:misr}
\end{figure}
%
Note that for such information to exist there should be sub-pixel\anote{subpixel} shifts in either the imaging system or the scene between consecutive images.


For typical imaging use-cases, high resolution images are preferable to low resolution images; higher resolutions are
desirable in and of themselves and as inputs to later image processing transformations that can degrade image quality (e.g.\ by virtue of quantization or compression).
%
In theory the resolving power of an imaging system is primarily determined by the number of independent sensor elements that comprise that imaging system (each of which collects a component of the ultimate image).
%
Naturally then, a way to increase the resolution of such a system is to increase the density of such sensor elements per unit area.
%
Unfortunately, and counter-intuitively, since the number of photons incident on each sensor decreases as the sensor shrinks, shot noise\anote{shotnoise} thwarts that idea.
%
Furthermore, while sensor density is primary, secondary effects due to optics limit resolution as well;
the point spread of a lens (distortion of a point source due to diffraction), chromatic aberrations (distortion due to differing indices of refraction for differing wavelengths of light), and motion blur all function to obscure or erase details from the image.


In domains such as satellite/aerial photography, medical imaging, and facial recognition,
high-resolution reconstruction of low-resolution samples is eminently useful since ab-initio acquisition of high-resolution images is either logistically difficult or impossible due to aforementioned imaging apparatus limitations.
%
For example in the instance of satellite imagery, acquisition of high-resolution imagery is primarily hampered by optics and physics\anote{satelliteoptics}.
%
In contrast, in the cases of medical imaging (where procedures are invasive and patient exposure time needs to be minimized\cite{doi:10.1002.cmr.a.21249}) and facial recognition (e.g.\ for purposes of surveillance) the primary challenge is logistics and access to repeat collection opportunities.


The benefits of enhancing images using SR techniques include not only more pleasing or more readily interpretable images for human consumption but higher quality inputs for automated learning systems as well.
%
In particular object detection systems trained on super-resolved images outperform those trained on the low
resolution originals\cite{effectssuperres}.
%
Indeed this is our ultimate goal - not super-resolution per se but super-resolution in the service of improved object detection performance for longwave-infrared (LWIR) imagery.
%
Note that while practically speaking, there exist hardware and software solutions for increasing the resolution of an imaging
system, we, owing to a "Ship of Theseus" consideration, discount such propositions.
%
We instead take low resolution images as given and seek techniques that allow for ex post facto reconstruction or inference of precise details.
%
This necessarily constrains techniques under consideration to be algorithmic in nature and software in practice.

The rest of this survey is outlined as follows: Section~\ref{sec:background} introduces imaging systems, notation, and the model of imaging that will be the mathematical framework for the proceeding sections, Section~\ref{sec:classical-algorithms} surveys classical techniques (those that do not employ neural networks), Section~\ref{sec:deep-learning-algorithms} surveys neural-network techniques with heavy emphasis on deep learning (i.e.\ deep networks), Section~\ref{sec:future-research} discusses the scope and goals of the author's research program, and Section~\ref{sec:conclusion} summarizes.


\section{Background}\label{sec:background}
\localtableofcontents
\section{Background}\label{sec:background}
\localtableofcontents
\subsection{Imaging systems}\label{subsec:imaging-systems}
\begin{figure}[!htbp]
	\includegraphics[width=\linewidth,keepaspectratio]{figures/background/bccd.png}
	\caption{CCD buried channel MOS capacitor \cite{finaltestguideline}.}
	\label{fig:mos-cap}
\end{figure}

We begin with a practical discussion of imaging systems.
%
An imaging system consists of an imaging sensor and optics. 
%
An imaging sensor is a device that converts an optical image into a digital signal.
%
Charge-coupled devices (CCD) and complementary metal-oxide-semiconductor (CMOS) devices are the most common imaging sensors; CCDs have better performance while CMOS devices are newer and less expensive.
%
A third type that's of particular interest to us is the microbolometer, which is used as a sensor in thermal cameras.

\subsubsection{CCD Devices}

CCDs consist of two components: an imaging component and a readout component.
%
The imaging component is arranged with every third stripe of electrode tied electrically to form three sets of equipotentials.
%
These equipotentials taken together constitute a \textit{vertical register} (VR) (see figure~\ref{fig:ccd-array}).
\begin{figure*}[!htbp]
	\centering
	\includegraphics[width=.8\textwidth,keepaspectratio]{figures/background/ccd_array.png}
	\caption{CCD array with both image and readout components \cite{pawley1995handbook}. Note that all electrodes intersecting \(\Phi i\) for some \(i\) are at the same potential (voltage).}
	\label{fig:ccd-array}
\end{figure*}
%
A vertical register moves the collected photoelectrons down one electrode line at a time, using charge coupling, with the aid of channel stops (which function to prevent diffusion of charge across channels).

More precisely the imaging component of a CCD consists of densely packed two-dimensional arrays of buried channel\anote{buriedchannel} metal-oxide-semiconductor (MOS) capacitors (see figure~\ref{fig:mos-cap}).
%
Individual MOS capacitors are biased by a gate voltage such that a potential well develops in the n-type\anote{ntype} silicon.
%
This potential well acts as a storage system for charge induced by the inner photoelectric effect\anote{innerphotoelectric}.
%
When photons are incident on a MOS capacitor some of the photons are absorbed at the surface, some are scattered at the surface, and some are transmitted into the bulk material.
%
Those photons that are transmitted interact with electrons in the valence band of the silicon substrate and thereby excite them into the semiconductor's conduction band.
%
This creates electron-hole pairs that either diffuse or recombine (in the silicon).
%
For high-quality silicon, the lifetime of such a pair is several milliseconds (before recombination) \cite{scientificccd}.
%
The electrons of the electron-hole pairs (that do not recombine) then diffuse into the potential well, while the holes migrate to the grounded substrate (i.e., out of the sensor).
%
Electrons created in this way are called \textit{photoelectrons}.

Suppose the VR is in a state such that there's a collection of photoelectrons on each channel at equipotential \(\Phi1\) and only \(\Phi1\). Note this means \(\Phi2, \Phi3\) are at \(0\)v (again just as in figure~\ref{fig:mos-cap}). The VR mechanism that shifts collected charge operates as follows:
\begin{framed}
\begin{enumerate}
	\item \(\Phi2\) is positively biased to \(10\)V. This causes redistribution of charge such that it is diffused underneath both \(\Phi1\) and \(\Phi2\).
	\item \(\Phi1\) is set to \(0\)v. This concentrates the charge exclusively under \(\Phi2\).
	\item The same redistribution is repeated vis-à-vis \(\Phi2, \Phi3\) and \(\Phi3, \Phi1\).
	\item The entire process repeats thereby shifting the charge three electrode lines (or one pixel row) at a time.
	\item At the bottom of the image section \(\Phi3\) transfers all charge to the horizontal register which functions much like the VR except faster.
\end{enumerate}
\end{framed}
%
An obvious challenge faced by this system is how to prevent errant charge from accumulating out of sync with the shift process, i.e., how to prevent new photoelectrons from being produced at intermediate electrode lines while far lines are being shifted.
%
The solution is to have interstitial dedicated shift channels in between columns of sensors, with the shift channels being masked off from exposure to light.
%
This type of reading is called \textit{interline transfer} because the accumulated charge is first moved one line over, into the shift channels.
%
Naturally, interline transfer shrinks photosensitive area by half and despite possible solutions (e.g, micro-lenses being used to focus most of the light onto the unmasked sensors) this is one of the drawbacks of CCDs relative to CMOS devices.

\subsubsection{CMOS Devices}

CMOS devices consist of arrays of photodiodes (see figure~\ref{fig:photodiode.png}).
\begin{figure}[!htbp]
    \centering
    \begin{subfigure}[b]{0.49\textwidth}
        \includegraphics[width=\linewidth,keepaspectratio]{figures/background/cmos/photodiode.png}
        \caption{Photodiode schematic. L\textsubscript{e}, L\textsubscript{h} are electron, hole diffusion lengths respectively \cite{Xu2015FundamentalCO}.}
        \label{fig:photodiode.png}
    \end{subfigure}
    \vskip\baselineskip
    \begin{subfigure}[b]{0.49\textwidth}
        \center
        \includegraphics[width=.8\linewidth,keepaspectratio]{figures/background/cmos/3t_pixel.png}
        \caption{Three transistor pixel. M\textsubscript{rst} is the reset transistor (enabling the photodiode to dump charge), M\textsubscript{sf} buffers the charge on the photodiode (so that it can be read without loss), and M\textsubscript{sel} enables a whole row of pixels to be read simultaneously (since all pixels in a physical row are tied to the same row line).}
        \label{fig:3tpixel}
    \end{subfigure}
    \caption{CMOS components.}
\end{figure}
A photodiode is a \textit{p-n junction}\anote{pnjunction} operated in reverse bias mode\anote{reversebias}.
%
When a photon of sufficient energy is absorbed by the diode, it creates an electron-hole pair (as a result of the photoelectron effect).
%
If the creation event happens within the active region then the hole migrates out through the p-type material and the electron migrates out through the n-type material.
%
This establishes a \textit{photocurrent} that can be interrogated by a reading circuit (see figure~\ref{fig:3tpixel}).

CMOS sensor arrays do not shift the charge from row to row like CCD arrays.
%
In a CMOS sensor array, each pixel contains a transistor M\textsubscript{sel} controlled by the voltage applied across a row (see figure~\ref{fig:cmosarray}).
%
In order to read one row of pixels, a rowline is raised high to turn on (close) all the M\textsubscript{sel} transistors in the row.
%
This brings the signals from all the pixels in that row to the shift register below by way of the column lines.
%
The shift register then outputs the values of the pixels.

The high number of transistors needed per pixel in CMOS arrays has only recently become feasible for semiconductor foundries to manufacture.
%
This, along with such artifacts as the rolling shutter effect produced by rowline reading, are some of the drawbacks of CMOS devices relative to CCD devices.
%
Despite this CMOS devices have become the most common imaging system in consumer goods such as cell phones and digital cameras due to their relatively simple mechanics.
\begin{figure}[!htbp]
	\center
	\includegraphics[width=.8\linewidth,keepaspectratio]{figures/background/cmos.png}
	\caption{CMOS array of red, green, and blue pixels.}
	\label{fig:cmosarray}
\end{figure}

\subsubsection{Microbolometers}

Despite their ubiquity in visible spectrum devices, neither CCD arrays nor CMOS arrays can be adapted to capture any portion of the infrared spectrum.
%
A microbolometer, on the other hand, measures the power in the infrared by exposing a thermistor\anote{thermistor} to the incident light.
%
\begin{figure}[!htbp]
	\center
	\includegraphics[width=\linewidth,keepaspectratio]{figures/background/microbolometer2.png}
	\caption{Bridge structure of Honeywell microbolometer \cite{KESIM2014245}.}
	\label{fig:microbolometer}
\end{figure}
%
Since a thermistor's resistance changes as a function of its temperature, a key issue in the design of a microbolometer is the thermal isolation of the thermistor.
%
With the maturation of micro-machining techniques (such as for MEMS\anote{mems} devices) over the last few years efficient microbolometers have become possible.
%
There are now many consumer thermal cameras on the consumer market powered by two-level microbolometer pixels consisting of a thermo-sensitive component suspended above (and insulated from) silicon (see figure~\ref{fig:microbolometer}).

These pixel packages are evacuated in order that they have good conduction, convection, and radiation heat transfer properties.
%
The actual thermo-sensitive component consists of a thermistor, an absorber (which aids in the transfer of heat to the thermistor), and a reflector that creates a Fabry–Pérot optical cavity\anote{fabryperot} (typically \({\sim}\lambda/4\) \cite{bolometer}) that traps the infrared light.
%
Typical materials for the thermistor are vanadium oxide and amorphous silicon owing to their high temperature coefficients of resistance \cite{bolometer}, which, in effect transform small changes in temperature into large changes in resistance.
%
Measurements of the thermistor are performed by a read-out integrated circuit adjacent to the bridge in the silicon substrate.
%
It is plain to see that microbolometers are designed much differently from either CCD or CMOS arrays.
%
As a result of this fact, high-resolution infrared imaging systems are not as of yet available on the consumer market and hence our interest in applying super-resolution to images collected by such systems.
%
With that in mind, we now proceed to formalizing the problem of super-resolution.

\subsection{Interlude: Mathematical Notation}\label{subsec:notation}

Upper case plain latin \(X, Y\) each denote channel \(\times\) row \(\times\) column \textit{tensors}\anote{tensor} representing LR and HR images respectively, with \((0, 0,0)\) corresponding to the top left corner of the first channel of image.
%
Often for the sake of simplicity we consider grayscale images, in which case we omit the channel dimension.
%
\(D, H, F, G\) variously refer to functions that operate on images.
%
Bolded uppercase latin are reserved for batches of objects.
%
For example \(\bm{X}, \bm{Y}\) are batches of images, i.e., batch size \(\times\) channel \(\times\) row \(\times\) column tensors with \((0, 0, 0,0)\) corresponding to the top left corner of the first channel of first image in the batch.
%
Bolded lower case latin such as \(\bm{x}, \bm{y}, \bm{z}\) denotes a conventional column or row vector.
%
Subscripts are used most often to indicate sequences (e.g., \(X_1, X_2, \dots\)) but occasionally indicate position (e.g., \((U_x, U_y)\)).
%
The \(L_2\) norm is indicated by unadorned \(\abs{\cdot}\).
%
All other norms are indicated by a subscript (e.g. \(\abs{\cdot}_1\) is the \(L_1\) norm).
%
New mathematical quantities are defined on first use by colon-equals \(\coloneqq\).
%
Hats denote computed estimators (e.g., \(\hat{X}\) is an estimate of \(X\) given noisy samples \(X_i\)).
%
All other notation is defined in situ.

\subsection{Imaging model}\label{subsec:imaging-model}

Figure~\ref{fig:bertrand} shows a conceptual model of the imaging process as carried out by an imaging system.
%
The input to the system is a natural scene that is in effect sampled by the imaging system.
%
In the idealized case, the sampling is done at (or above) the Nyquist rate and no aliasing occurs.
%
In practice there is noise and loss introduced at every step of the process: atmospheric turbulence plays a role at large distances, motion produces multiple views of the same scene but also induces blur, imperfections of the lenses further blur the image, and finally down-sampling by the sensor elements into pixels produces aliasing artifacts\anote{ccd}.
%
The noisy, blurry, down-sampled images are then further degraded by sensor noise.
%
Each such image we call an LR sample.
\begin{figure*}
	\centering
	\begin{adjustbox}{width=\textwidth}
		\begin{tikzpicture}[auto]
			\tikzstyle{terminal} = [rectangle, draw, text width=5em, text centered, minimum height=4em]
			\tikzstyle{block} = [rectangle, draw, fill=gray!20, text width=6em, text centered, rounded corners, minimum height=4em]
			\tikzstyle{line} = [draw, -latex']
			\tikzstyle{sum} = [circle, draw]

			\node[inner sep=0pt] (bertrand) {\includegraphics[width=.15\textwidth]{figures/bertrand/bertrand.png}};
			\node [above = of bertrand] (scene) {Scene};

			\node[sum, right = of bertrand] (sum1) {$+$};
			\node [block, below = of sum1] (atmo-noise) {Atmospheric noise};

			\node [block, right = of sum1] (motion) {Translation, Rotation, Aspect};
			\node [above = of motion] {Motion};

			\node [inner sep=0pt, right = of motion] (motion-output) {};

			\node[inner sep=0pt, below = of motion-output] (bertrand-motion) {\includegraphics[width=.15\textwidth]{figures/bertrand/bertrand.png}};
			\node[inner sep=0pt, below = of motion-output, xshift=2mm, yshift=-2mm] {\includegraphics[width=.15\textwidth]{figures/bertrand/bertrand.png}};
			\node[inner sep=0pt, below = of motion-output, xshift=4mm, yshift=-4mm] {\includegraphics[width=.15\textwidth]{figures/bertrand/bertrand.png}};
			\node[inner sep=0pt, below = of motion-output, xshift=6mm, yshift=-6mm] {\includegraphics[width=.15\textwidth]{figures/bertrand/bertrand.png}};

			\node [block, right = of motion-output] (blur) {Optical, Motion};
			\node [above = of blur] {Blur};
			\node [inner sep=0pt, right = of blur] (blur-output) {};

			\node[inner sep=0pt, below = of blur-output] (bertrand-blur) {\includegraphics[width=.15\textwidth]{figures/bertrand/bertrand-blur.png}};
			\node[inner sep=0pt, below = of blur-output, xshift=2mm, yshift=-2mm] {\includegraphics[width=.15\textwidth]{figures/bertrand/bertrand-blur.png}};
			\node[inner sep=0pt, below = of blur-output, xshift=4mm, yshift=-4mm] {\includegraphics[width=.15\textwidth]{figures/bertrand/bertrand-blur.png}};
			\node[inner sep=0pt, below = of blur-output, xshift=6mm, yshift=-6mm] {\includegraphics[width=.15\textwidth]{figures/bertrand/bertrand-blur.png}};

			\node[block, right = of blur-output] (downsample) {Quantization, Pixel-binning};
			\node [above = of downsample] {Down-sampling};

			\node[sum, right = of downsample] (sum2) {$+$};
			\node [block, below = of sum2] (sensor-noise) {Sensor noise};

			\node[inner sep=0pt, right = of sum2] (bertrand-blur-noise) {\includegraphics[width=.1\textwidth]{figures/bertrand/bertrand-blur-noise.png}};
			\node[inner sep=0pt, right = of sum2, xshift=2mm, yshift=-2mm] {\includegraphics[width=.1\textwidth]{figures/bertrand/bertrand-blur-noise.png}};
			\node[inner sep=0pt, right = of sum2, xshift=4mm, yshift=-4mm] {\includegraphics[width=.1\textwidth]{figures/bertrand/bertrand-blur-noise.png}};
			\node[inner sep=0pt, right = of sum2, xshift=6mm, yshift=-6mm] {\includegraphics[width=.1\textwidth]{figures/bertrand/bertrand-blur-noise.png}};
			\node [above = of bertrand-blur-noise, text width=5em] (lr-output) {Low-resolution images};

			\draw[-] (bertrand) edge (sum1);
			\draw[->] (sum1) edge (motion);
			\draw[->] (atmo-noise) edge (sum1);

			\draw[->] (motion) edge (blur);
			\draw[->] (motion-output) edge (bertrand-motion);

			\draw[->] (blur) edge (downsample);
			\draw[->] (blur-output) edge (bertrand-blur);
			\draw[->] (sensor-noise) edge (sum2);
			\draw[-] (downsample) edge (sum2);
			\draw[->] (sum2) edge (bertrand-blur-noise);
		\end{tikzpicture}
	\end{adjustbox}
	\caption{The imaging model illustrating the relationship between a scene and final low-resolution images due to noise, motion, blur, and sampling.}
	\label{fig:bertrand}
\end{figure*}

Let \(Y\) denote an idealized HR image of the scene from some fixed vantage point and assume the imaging system collects \(K\) LR samples \(\Xk\) of \(Y\).
%
Formally the \(\Xk\) are related to \(Y\) by
\begin{equation}
	\Xk = (D \circ H_k \circ A_k) (Y) + \varepsilon_k
	\label{eqn:imagingmodel}
\end{equation}
where for the \(k\)th LR sample, \(A_k\) (of \(K\)) is the operator representing motion, \(H_k\) represents the blur operator, \(D\) represents the down-sampling operator (constant in time since it's typically a digital component of the imaging system), and \(\varepsilon_k\) represents the composite noise (environment and sensor noise).

\subsubsection{Motion}

In the case of MISR, high-resolution images are constructed by exploiting distinct scene data among multiple low-resolution images.
%
Such distinct data are produced by the relative motion of the imaging system and the scene and therefore precise and accurate image \textit{registration}\anote{registration} is paramount.
%
Successful registration is effected by producing a transformation \(f\) that maps pixels at coordinates \((x',y')\) in sample \(\Xkone\) to coordinates \((x,y)\) in sample \(X_j\)
\begin{equation}
	\Xkone(f(x', y')) = X_j(x,y)
	\label{eqn:registration}
\end{equation}
where we use index \(j\) to indicate the possibility of registering all samples relative to a fixed reference image (e.g. \(j=0\) the first sample) or successive/relative registration (i.e.,with \(j=k\)).
%
While image registration is already challenging in the various domains where samples are treated as canonical, such as remote sensing and medical imaging, in SR it is further complicated by the fact that the images to be registered are assumed uncertain.
%
Therefore, in principle, image registration and super-resolution cannot be decoupled, since image registration accuracy would be improved by operating on the estimated HR images;
%
Hardie \etal \cite{Hardie1997} use a Bayesian framework to jointly estimate image registration parameters and the high-resolution image \cite{Hardie1997}.
%
Alternatively one can marginalize over the HR image and estimate the registration parameters using maximum-likelihood estimation (MLE) (see section~\ref{subsubsec:gaussianprocess}).

\subsubsection{Blur}

We now consider the challenges and nuances of estimating blur.
%
In general, the optical transfer function (OTF) characterizes the blur of an imaging system\anote{otf}.
%
We factor the OTF into three components:
\begin{equation}
	H(u, v) \coloneqq H_{\text{diff}}(u,v) H_{\text{abr}}(u,v) H_{\text{int}} (u,v)
\end{equation}
where \(u,v\) are horizontal and vertical spatial frequencies respectively (measured in cycles/mm), \(H_{\text{diff}}\) is blur due to diffraction, \(H_{\text{abr}}\) is blur due to lens aberrations, and \(H_{\text{int}}\) is blur due to imaging sensor shape (obtained by taking the Fourier transform of the shape an individual sensor in the imaging array).
%
Blur due to diffraction in most imaging systems is due to diffraction through a circular aperture \cite{goodman2005introduction}:
\begin{equation*}
	H_{\text{diff}}(u,v) =   \begin{cases}
		\frac{2}{\pi} \left(\frac{1}{\cos(\tau)} - \tau \sqrt{1-\tau^2}\right) & \text{if } \tau < 0 \\
		0                                                                      & \text{otherwise}
	\end{cases}
\end{equation*}
where \(\tau = \rho/\rho_c\), \(\rho=\sqrt{u^2 +v^2}\), \(\rho_c = 1/\lambda N\) is the radial cutoff frequency of the aperture, \(N\) is the f-number\anote{fnumber} of the optics, and \(\lambda\) is the wavelength of light being diffracted.
%
This is in fact the filter that produces the Airy pattern and therefore informs sensor array spacing in order to avoid aliasing.
%
Wavelength independent blurring due to aberrations can be induced by various imperfections in the lenses such as spherical aberration, comatic aberration\anote{coma}, or astigmatism. Furthermore, dispersion\anote{dispersion} blurs particular wavelengths of light. A good model for all of these effects is \cite{10.1117.12.946501}:
\begin{equation*}
	H_{\text{abr}}(u,v) =   \begin{cases}
		1-\left(\frac{25}{65}\right)^2 \left(1-4\left(\tau - \frac{1}{2}\right)\right)^2 & \text{if } \tau < 0 \\
		0                                                                                & \text{otherwise}
	\end{cases}
\end{equation*}
Figure~\ref{fig:mtf} shows an example OTF for an imaging system with a sensor spacing of 0.050 mm and therefore sampling frequency of 20 cycles/mm and \(\rho_c = 83.3~\text{cycles}/\text{mm}\) (\(F=3\) and \(\lambda = 4\mu\text{m}\) i.e.,near infrared).
%
Notice that \(\rho_c\) is much greater than the Nyquist rate (\(\frac{1}{2} \times 20~\text{cycles}/\text{mm} = 10~\text{cycles}/\text{mm}\)) and therefore many frequencies that are within the radial cutoff frequency will be aliased.
%
This in particular can be mitigated by effectively increasing sampling rate using MISR.
%
Notice also that like a typical transfer function the OTF is not flat and therefore attenuates high spatial frequencies.
%
Simply applying gain to the image wouldn't solve the attenuation problems because of aliasing, but likewise this can be resolved after the effective sampling rate is increased using MISR.
\begin{figure}[!htbp]
	\includegraphics[width=\linewidth,keepaspectratio]{figures/background/mtf.png}
	\caption{OTF magnitude cross-section for \cite{milanfar2017super}.}
	\label{fig:mtf}
\end{figure}

The challenge of super-resolution is to solve the inverse problem of finding \(Y\) from one or several \(\Xk\).
%
In general, since \(A_k, H_k, D_k\) are highly degenerate functions, the corresponding inverse problems are ill-posed without regularization and conditioning.
%
The techniques that have been brought to bear on the problem range from interpolation to statistical estimation to example based learning.


\section{Image Registration}\label{sec:image-registration}
\localtableofcontents
\begin{figure*}
    \begin{tikzpicture}[scale=1,every node/.style={minimum size=1cm},on grid]
        \tikzstyle{diff} = [draw, circle, scale=0.5]
        \foreach \Sigma in {0,...,4} {
            \node[yshift=-350+\Sigma*30, yslant=0.5, xslant=-1.5, scale=1] (bertrand-blur-\Sigma-1) {
                \includegraphics[width=.15\textwidth]{figures/sift/bertrand_sigma_1_\Sigma.png}
            };

        }
        \foreach \Sigma in {0,...,4} {
            \node[yshift=-150+\Sigma*20, xshift=30, yslant=0.5, xslant=-1.5, scale=0.5] (bertrand-blur-\Sigma-2) {
                \includegraphics[width=.15\textwidth]{figures/sift/bertrand_sigma_1_\Sigma.png}
            };
        }
        \foreach \Sigma in {0,...,4} {
            \node[yshift=-10+\Sigma*20, yslant=0.5, xshift=30, xslant=-1.5, scale=0.25] (bertrand-blur-\Sigma-3) {
                \includegraphics[width=.15\textwidth]{figures/sift/bertrand_sigma_1_\Sigma.png}
            };
        }

        \foreach \Sigma in {1,...,4} {
            \begin{scope}[
               yshift=-365+\Sigma*30, xshift=280,every node/.append style={
               yslant=0.5, xslant=-1.5},yslant=0.5, xslant=-1.5
           ]
               \node[] (bertrand-dog-\Sigma-1) {
                    \includegraphics[width=.15\textwidth]{figures/sift/dog_bertrand_sigma_1_\Sigma.png}
                };
               \fill[green] (2.05,2.05) rectangle (2.35,2.35); % center pixel
               \fill[green] (1.65,2.05) rectangle (1.95,2.35); %left
               \fill[green] (2.45,2.05) rectangle (2.75,2.35); %right
               \fill[green] (2.05,2.45) rectangle (2.35,2.75); %top
               \fill[green] (2.05,1.95) rectangle (2.35,1.65); %bottom
            \end{scope}

%            \node[] (bertrand-dog-\Sigma-1) {
%                \includegraphics[width=.15\textwidth]{figures/sift/dog_bertrand_sigma_1_\Sigma.png}
%            };
           \draw[yshift=-365+\Sigma*30, xshift=280, step=1mm, black, yslant=0.5, xslant=-1.5] (-1,-1) grid (0,0);
           \draw[yshift=-365+\Sigma*30, xshift=280, black,thick, yslant=0.5, xslant=-1.5] (-1,-1) rectangle (0,0);
        }
        \foreach \Sigma in {1,...,4} {
            \node[yshift=-365+\Sigma*30, xshift=140, diff] (diff-\Sigma-1){\Large$-$};
        }



        \foreach \Sigma in {1,...,4} {
            \node[yshift=-160+\Sigma*20, xshift=250, yslant=0.5, xslant=-1.5, scale=0.5] (bertrand-dog-\Sigma-2) {
                \includegraphics[width=.15\textwidth]{figures/sift/dog_bertrand_sigma_1_\Sigma.png}
            };
        }
        \foreach \Sigma in {1,...,4} {
            \node[yshift=-160+\Sigma*20, xshift=140, diff] (diff-\Sigma-2){\Large$-$};
        }



        \foreach \Sigma in {1,...,4} {
            \node[yshift=-5+\Sigma*20, xshift=250, yslant=0.5, xslant=-1.5, scale=0.25] (bertrand-dog-\Sigma-3) {
                \includegraphics[width=.15\textwidth]{figures/sift/dog_bertrand_sigma_1_\Sigma.png}
            };
        }
        \foreach \Sigma in {1,...,4} {
            \node[yshift=-5+\Sigma*20, xshift=140, diff] (diff-\Sigma-3){\Large$-$};
        }

        % draw annotations
        \foreach \x in {1,2,3} {
            \foreach \Sigma in {1,...,4} {
                \draw[-] (bertrand-blur-\Sigma-\x) edge (diff-\Sigma-\x);
            }
            \foreach \Sigma in {0,...,3} {
                \pgfmathtruncatemacro{\nodelabel}{1+\Sigma}
            \draw[-] (bertrand-blur-\Sigma-\x) edge (diff-\nodelabel-\x);
            }
            \foreach \Sigma in {1,...,4} {
                \draw[->] (diff-\Sigma-\x) edge (bertrand-dog-\Sigma-\x);
            }
        }


%           \fill[green] (2.05,2.05) rectangle (2.35,2.35); % center pixel
%           \fill[green] (1.65,2.05) rectangle (1.95,2.35); %left
%           \fill[green] (2.45,2.05) rectangle (2.75,2.35); %right
%           \fill[green] (2.05,2.45) rectangle (2.35,2.75); %top
%           \fill[green] (2.05,1.95) rectangle (2.35,1.65); %bottom
%    % 8 -pixel setting
%           \fill[green] (1.65,2.45) rectangle (1.95,2.75); %top-left
%           \fill[green] (2.45,2.45) rectangle (2.75,2.75); %top-right
%           \fill[green] (2.75,1.95) rectangle (2.45,1.65); %bottom-right
%           \fill[green] (1.65,1.95) rectangle (1.95,1.65); %bottom-left
%    % 2. ring
%           \fill[green] (1.25,1.55) rectangle (1.55,1.25); %bottom-left
%           \fill[green] (0.85,1.55) rectangle (1.15,1.25); %bottom-left
%           \fill[green] (0.85,1.15) rectangle (1.15,0.85); %bottom-left
%           \fill[green] (1.25,0.75) rectangle (1.55,0.45); %bottom-left


%    \draw[-latex,thick,blue](-3,5)node[left]{ }
%    to[out=0,in=90] (0.8,1.15);
%    \draw[-latex,thick,green](-3,5)node[left]{3 patches}
%    to[out=0,in=90] (0,2.8);
%    %
%    \draw[-latex,thick,green](-3,-2)node[left]{1 patch}
%    to[out=0,in=200] (-1,-.9);
%    \draw[thick,gray!70!black](6,4) node {4 neighbourhood rule};
%    \draw[thick,gray!70!black](6,-2) node {8 neighbourhood rule};
    %
    \end{tikzpicture}
%\end{adjustbox}
\caption{The imaging model illustrating the relationship between a scene and final low-resolution images due to noise, motion, blur, and sampling.}
\label{fig:bertrand}
\end{figure*}

Images obtained from multiple vantage points, or at different times, of the same scene, become distorted with respect
to each other.
%
Since in MISR the aim is to exploit new information across multiple LR samples we need to first rectify these distortions and reconcile the images.
%
Effectively this means finding one or more pixel transformations that enable mapping all LR images to a common pixel grid.
%
When the transformations cannot be deduced from first principles (e.g. precise knowledge of the relative motion of the scene and the imaging system) they must be estimated from the LR images.
%
There are broadly two perspectives on constructing the transformation \(f\) (see eqn.~\eqref{eqn:registration}): the global perspective which aims to model motion as a map of the image as a whole and the local perspective which aims to model motion as a deformation of individual pixels.
%
These two perspectives naturally correspond to a globally or locally defined transformation \(f\).
%
We first cover global algorithms, examining the different techniques available for each step, and then move on to local algorithms.

\subsection{Global Algorithms}

Most global image registration algorithms consist of a feature detection and selection step (also called Control Point (CP) selection), a feature matching step, and a transform estimation step.
%

\subsubsection{Feature Detection and Selection}

Feature detection and selection is the process of identifying features of the image that are presumed to be invariant across the multiple images to registered.
%
Note that here by features we mean image artifacts (e.g. edges, contours, line intersections, or corners); in this context encodings or transformations of these image artifacts are called \textit{descriptors}.
%
The CPs are the data that will be used to estimate the transformation \(f\).
%
Therefore, in order that the estimated transformation is accurate, CPs should be robust to noise and image degradation, sufficiently distributed throughout the image, and readily matched in the matching step.

\paragraph{Harris Corner Detection}
Bentoutou \etal\cite{bentoutou2005automatic} use a Harris detector\cite{harris1988combined} to find corner points, arguing that corners are robust to noise and stable over multiple images.
%
The Harris detector improves on the Moravec\cite{moravec1980obstacle} detector.
%
The Moravec detector starts from the error function \(E_{x,y}(u,v)\) which computes the sum of the squared differences (SSD) between an \(m \times m\) weighted window around a pixel \(X(x, y)\) and weighted windows shifted by \(u,v\) pixels:
\begin{multline}
    \quad E_{x,y}(u,v) \coloneqq \\ \sum_{i,j=-m/2}^{m/2} w_{ij}\left[ X(x_i+ u,y_j+v) - X(x_i, y_j)\right]^2
    \label{moravecerrorfunction}
\end{multline}
where \(x_i \coloneqq x + i\) and \(y_j \coloneqq y+j\).
\begin{figure}
    \centering
    \includegraphics[width=\linewidth,keepaspectratio]{figures/registration/corners.png}
    \caption{Moravec Corner Detector}
    \label{fig:corners}
\end{figure}
Moravec assigns a "corner score" according to the following reasoning (see figure~\ref{fig:corners}):
\begin{enumerate}
    \item If a pixel is in a region of uniform intensity then \(E_{x,y}(u,v)\) is small for all \(u,v\) (since neighboring windows are similar).
    \item If a pixel is on an edge, then \(E_{x,y}(u,v)\) for either \(u > 0\) or \(v > 0\), but not both, is high.
    \item If a pixel is on a corner, then \(E_{x,y}(u,v)\) for \(u > 0\) and \(v > 0\) is high.
\end{enumerate}
Therefore the corner score at pixel coordinate \((x,y)\) is \(\min_{u,v} E_{x,y}(u,v)\) in order to select for the third case.
%
Moravec comments that this corner score is not isotropic, i.e. if edges aren't aligned with either the pixel axes or diagonals then \(E_{x,y}(u,v)\) will incorrectly be low.
%
Harris' insight was to linearize \(E_{x,y}(u,v)\) in order to compute a quantity more closely related to the intensity variation in a local
neighborhood of a pixel:
\begin{equation}
    X(x_i + u,y_j + v) \approx  X(x_i,y_j) + \frac{\partial X}{\partial u}u + \frac{\partial X }{\partial v} v
\end{equation}
where the partial derivatives are taken at \((x,y)\).
%
This implies
\begin{align}
    E_{x,y}(u,v) & \approx \sum_{i,j=-m/2}^{m/2} w_{ij} \left[ X_u u + X_v v\right]^2                 \\
                 & = \sum_{i,j=-m/2}^{m/2} w_{ij} \left[ X_u^2 u^2 + X_v^2 v^2 + 2 X_u X_v u v\right] \\
                 & = \left[ u,v \right] \begin{bmatrix}
        \sum w_{ij}X_u^2   & \sum w_{ij}X_u X_v \\
        \sum w_{ij}X_u X_v & \sum w_{ij}X_v^2   \\
    \end{bmatrix}  \begin{bmatrix}
        u \\
        v
    \end{bmatrix}          \\
                 & = \left[ u,v \right] M  \begin{bmatrix}
        u \\
        v
    \end{bmatrix} \label{eqn:structurematrix}
\end{align}
where \(X_u = \partial X/\partial u\) and similarly \(X_v\).
%
The matrix in eqn.~\eqref{eqn:structurematrix}, called the \textit{structor tensor} or \textit{second-moment matrix} \(M\), is the quantity Harris investigated.
%
Harris reasoned that the cases of Moravec correspond to conditions on the eigenvalues \(\lambda_1, \lambda_2\) of \(M\):
\begin{enumerate}
    \item If \(\lambda_1 \approx \lambda_2 \approx 0\) then \(X(x,y)\) is in a region of uniform intensity.
    \item If \(\lambda_1 \gg \lambda_2\) or \(\lambda_2 \gg \lambda_1\) then \(X(x,y)\) is on an edge.
    \item \(\lambda_1 \approx \lambda_2 > 0\) then \(X(x,y)\) is on a corner.
\end{enumerate}
Notice that if \(w_{ij} = 1\) then this is just the gradient covariance of the image and the Harris detector is essentially a local Principle Components Analysis (PCA).
%
In fact Harris doesn't actually compute the eigenvalues but instead a related quantity called the "strength":
\begin{align}
    S & = \lambda_1 \lambda_2 - \kappa (\lambda_1 + \lambda_2)^2 \\
      & = \det(M) - \kappa \operatorname{trace}^2(M)
    \label{eqn:strength}
\end{align}
%
Hence Bentoutou \etal~first compute a gradient map of the image using a first order Gaussian derivative filter.
%
They then threshold\anote{threshold} the gradient map at the average gradient value, thereby extracting only sufficiently "interesting" regions, and compute the strength \(S\) for all pixels.
%
They also apply Non-maximum Suppression\anote{nms} (NMS) using a \(3 \times 3\) window and further threshold the remaining non-zero strength values at a threshold of 1\% of maximum observed strength.
%
Finally only the "strongest" \(n\) corners are kept.

\paragraph{SIFT}

\begin{figure}
    \centering
    \includegraphics[width=\linewidth,keepaspectratio]{figures/registration/sift/sift_scale_invariant.png}
    \caption{Harris Detector failing to recognize the right image as a corner.}
    \label{fig:sift_harris}
\end{figure}
One issue with Harris detectors is that they're not invariant to scale (see figure~\ref{fig:sift_harris}).
%
Zahra \etal\cite{zahrasift} resolves this issue by using the Scale Invariant Feature Transform\cite{lowe2004distinctive} (SIFT) to identify CPs that, as the name implies, are invariant across multiple scales.
%
SIFT identifies scale invariant and noise robust features of an image, called \textit{keypoints}, by first finding candidate points with high local curvature at multiple scales and then culling according to some heuristics.
\begin{figure*}
	\tikzstyle{every picture}+=[remember picture]
	\tikzset{
		pics/greensquare/.style args={#1/#2/#3}{
				code = {
						\draw[green] (#1,#2) rectangle (#1+#3, #2+#3);
					}
			},
		subgradbin/.pic={
				\foreach \i in {0.0, 1.0} {
						\pgfmathsetmacro{\x}{0.5+\i*2*.08825};
						\foreach \j in {0.0, 1.0} {
								\pgfmathsetmacro{\y}{0.5+\j*2*.08825};
								\pic[] {greensquare=\x/\y/2*.08825};
							}
					}
			},
		subgradbins/.pic={
				\foreach \x in {0,1,2,3} {
						\pgfmathtruncatemacro{\angle}{45+\x*90}
						\pic[rotate around={\angle:(0.5,0.5)}] {subgradbin};
					}
			},
		pics/edgehistogram/.style args={#1/#2}{
				code={
						\begin{axis}[area style, width=2.5cm,height=2.5cm, hide axis, at={(#1cm,#2cm)}]
							\addplot+[ybar interval] plot coordinates {
									(-0.50, 2) (0.5, 4) (1.5, 5) (2.5, 3) (3.5, 2) (4.5, 2) (5.5, 0)
								};
							\path
							\foreach[count=\i from 0] \v in {2, 4, 5, 3, 2, 2} {
									(\i, \v) node[below] {\tiny\v}
								};
						\end{axis}
					}
			},
		pics/randomedgehistogram/.style args={#1/#2}{
				code={
						\begin{axis}[area style, width=2.5cm,height=2.5cm, hide axis, at={(#1cm,#2cm)}]
							\pgfmathrandominteger{\na}{3}{4}
							\pgfmathrandominteger{\nb}{3}{4};
							\pgfmathrandominteger{\nc}{3}{8};
							\pgfmathrandominteger{\nd}{3}{8};
							\pgfmathrandominteger{\ne}{3}{4};
							\pgfmathrandominteger{\nf}{3}{4};
							\pgfmathrandominteger{\ng}{3}{4};
							\pgfmathrandominteger{\nh}{3}{4};
							\addplot+[ybar interval] plot coordinates {
									(-0.50, \na)
									(0.5, \nb)
									(1.5, \nc)
									(2.5, \nd)
									(3.5, \ne)
									(4.5, \nf)
									(5.5, \ng)
									(6.5, \nh)
									(7.5, 0)
								};
							\path
							\foreach[count=\i from 0] \v in {\na, \nb, \nc, \nd, \ne, \nf, \ng, \nh} {
									(\i, \v) node[below] {\scalebox{.3}{\v}}
								};
							\coordinate (hist-#1#2) at (5,-3);
						\end{axis}
					}
			},
		array/.style={draw, minimum width=2em, minimum height=2em,
				outer sep=0pt},
	}
	\newcommand*{\siftwidth}{.5\linewidth}
	\newcommand*{\length}{sqrt((2.*x^2+2.*y^2)^2 + (8.*x^2*y^2)^2 )}
	\pgfplotsset{ % Define a common style, so we don't repeat ourselves
		dominantorientationaxis/.style={
				enlargelimits = false ,
				view={0}{90},
				xmin=0, xmax=1, ymin=0, ymax=1,
				ytick distance=1/16,
				xtick distance=1/16,
				axis equal image, grid=both,
				minor grid style={black},
				major grid style={black},
				axis equal image,
				samples=16
			},
		dominantorientationvectors/.style={
				domain=1/32:31/32,
				black,
				quiver={
						u={(2.*x^2+2.*y^2)/(2*\length)},
						v={8.*x^2*y^2/(2*\length)},
						scale arrows=0.2},
				-latex
			}
	}
	\pgfplotsset{ticks=none}
	\begin{subfigure}{\siftwidth}
		\centering
		\begin{tikzpicture}
			\begin{axis}[dominantorientationaxis]
				\addplot3[dominantorientationvectors] {0};
			\end{axis}
		\end{tikzpicture}
		\caption{First subfigure} \label{fig:1a}
	\end{subfigure}
	\begin{subfigure}{\siftwidth}
		\centering
		\begin{tikzpicture}[]
			\begin{axis}[dominantorientationaxis]
				\addplot3[dominantorientationvectors] {0};
				\draw (0.5, 0.5) circle [blue, radius=.5];
                \pic[rotate around={45:(0.5,0.5)}] {greensquare=.1464/.1464/.707};
				\draw[line width=2pt,red,-latex](.5,.5)--(1,1);
			\end{axis}
		\end{tikzpicture}
		\caption{First subfigure} \label{fig:1b}
	\end{subfigure}
	\vskip\baselineskip
	\begin{subfigure}{\siftwidth}
		\centering
		\begin{tikzpicture}[rotate=-45,transform shape]
			\clip[](2.04,2) rectangle (6.07,6.03);
			\begin{axis}[dominantorientationaxis, rotate around={45:(.5,.5)}, grid=none]
				\addplot3[dominantorientationvectors] {0};
				\pic[] {subgradbins};
			\end{axis}
		\end{tikzpicture}
		\caption{First subfigure} \label{fig:1c}
	\end{subfigure}
	\begin{subfigure}{\siftwidth}
		\centering
		\begin{tikzpicture}
			\coordinate (origin) at (0,0);
			\pgfmathtruncatemacro{\kmax}{div(3.0,1.0)}
			\begin{scope}[rotate=-45,transform shape,local bounding box=scope1]
				\foreach \i [evaluate=\i as \x using \i*1.0] in {0,...,\kmax}{
						\foreach \j [evaluate=\j as \x using \j*1.0] in {0,...,\kmax}{
								\pic {greensquare=\i/\j/1.0};
								\pgfmathtruncatemacro{\n}{\i*4+\j};
								\pic {randomedgehistogram=\i/\j};
							}
					}
			\end{scope}
			\begin{scope}[shift={($(scope1.south)-(0,1cm)$)}]
				\matrix (A) [matrix of math nodes, nodes={array, anchor=center}, column sep=-\pgflinewidth]
				{d_0,\dots,d_7 & d_8,\dots,d_{15} & \cdots\cdots\cdots & d_{120},\dots,d_{127} \\};
				\draw[decorate,decoration={brace, amplitude=10pt, raise=5pt, mirror}]
				(A-1-1.south west) to node[black,midway,below= 15pt] {$16 \times 8$ entries} (A-1-4.south east);%
				\path[-latex,red,thick] (hist-00) edge [] (A-1-1);
				\path[-latex,red,thick] (hist-10) edge [] (A-1-2);
				\path[-latex,red,thick] (hist-20) edge [] (A-1-3);
				\path[-latex,red,thick] (hist-30) edge [] (A-1-3);
				\foreach \i in {1,...,2}{
						\foreach \j in {0,...,3}{
								\path[-latex,red,thick] (hist-\j\i) edge [] (A-1-3);
							}
					}
				\path[-latex,red,thick] (hist-03) edge [] (A-1-3);
				\path[-latex,red,thick] (hist-13) edge [] (A-1-3);
				\path[-latex,red,thick] (hist-23) edge [] (A-1-3);
				\path[-latex,red,thick] (hist-33) edge [] (A-1-4);
			\end{scope}
		\end{tikzpicture}
		\begin{tikzpicture}

		\end{tikzpicture}
		\begin{tikzpicture}[overlay]
		\end{tikzpicture}
		\caption{First subfigure} \label{fig:1d}
	\end{subfigure}
	%	\caption{A figure that contains three subfigures} \label{fig:1}
\end{figure*}


%
It then "describes" these keypoints by a rotation invariant and noise robust representation.
%
The algorithm consists of five steps:
\begin{enumerate}
    \item Scale-space pyramid construction: a sequence of increasingly sub-sampled and more strongly Gaussian filtered images is computed. The sequence of differences of these images is also computed; the sequence of differenced images approximates the multi-scale Laplacian of Gaussians\anote{log} (LoG) of the image (see figure~\ref{fig:siftpyramid}).
    \item Keypoint detection: candidate keypoints are points on edges with curvature, i.e. extrema along scale and space dimensions in the LoG pyramid (see figure~\ref{fig:siftpyramid}).
    \item Keypoint selection: candidate keypoints are more precisely localized using an iterative process. Keypoints of low-contrast (therefore sensitive to noise) or on edges of low curvature\anote{smallcurvature} are culled.
    \item Keypoint orientation assignment: orientation is assigned to each keypoint by taking a weighted majority vote of all gradient orientations in a neighborhood of the keypoint (see figure~\ref{fig:siftdescriptorb}). Large minority votes (80\% of majority) are used to create more keypoints at the same pixel point.
    \item Keypoint descriptor computation: for each keypoint the descriptor is computed by partitioning the keypoint's neighborhood into \(2^k\) sub-neighborhoods, computing an 8-bin histogram of oriented gradients\anote{hog} (HOG) in each sub-neighborhood, and concatenating (see figure~\ref{fig:siftdescriptord}). In Lowe \etal\cite{lowe2004distinctive} \(2^4=16\) sub-neighborhoods are used to produce an \(8\times16 = 128\) entry length descriptor. The descriptor is also normalized to unit length in order to make it invariant to luminance (intensity).
\end{enumerate}
SIFT is indeed effective as a CP detector but unfortunately it is patented.
%
Alternatives include Binary Robust Invariant Scalable Keypoints\cite{leutenegger2011brisk}, and Oriented FAST and rotated BRIEF\cite{rublee2011orb} (which itself consists of applying Features from accelerated segment test\cite{rosten2006machine} to detect points of interest and Binary Robust Independent Elementary Features\cite{calonder2010brief} to compute descriptors).

\subsubsection{Feature Matching}

After robust features are identified in the reference image and the displaced images, they need to be matched.
%
For example for SIFT, where the descriptors are designed to be invariant across images, Euclidean distance using a \(k\)-d tree\anote{kdtree} can be used to efficiently match keypoint descriptors.
%
Although this often leads to false-positive matches (Zahra \etal~resolve this by using Random Sample Consensus (RANSAC)\anote{ransac}) it's a natural feature matching method.
%
In other cases the matching mechanism is not so straightforward; for a class of algorithms called area-based or intensity-based algorithms, that in fact combine the feature detection and matching step into one, matching involves comparing summaries of patches in the reference image and the displaced image.

\paragraph{Normalized Cross-correlation}
The simplest strategy for matching by region is to grid-search\anote{gridsearch} the space of possible translational shifts \(\Delta x, \Delta y\) and assess the quality of the registration for a given shift using a similarity metric.
%
One such similarity metric is Normalized Cross-correlation (NCC); for two image patches \(X, Y\) (with the same width, height) NCC is defined in a straightforward way
\begin{multline}
    NCC(X, Y) = \\ \frac{\sum_{x,y} \left(X(x,y) - \hat{X}\right) \left(Y(x,y) - \hat{Y}\right)}{\sqrt{\sum_{x,y} \left(X(x,y) - \hat{X}\right)^2} \sqrt{ \sum_{x,y} \left(Y(x,y) - \hat{Y}\right)^2 }}
\end{multline}
where \(\hat{X}, \hat{Y}\) are mean patch values.

\begin{figure}[!htbp]
	\begin{subfigure}{\linewidth}
		\centering
		\begin{adjustbox}{width=.8\textwidth}
			\begin{tikzpicture}
				\node[inner sep=0pt, xshift=50, yshift=-50] (shiftedcat) {
					{\transparent{.5}\includegraphics[]
							{figures/registration/cross-correlation/trimmed_cat.jpg}
						}};
				\node[inner sep=0pt, xshift=-50pt, yshift=50pt] (shiftedcat) {
					{\transparent{0.5}\includegraphics[]
							{figures/registration/cross-correlation/shifted_trimmed_cat.jpg}
						}};
				\draw[red, line width=1mm] (-250pt,-250pt) rectangle (250pt,250pt);
				\node(corrwin) at (0,250pt) {};
				\node[label={[label distance=1cm]:\scalebox{5}{Fixed cross-correlation window}}] (corrw) at (0,600pt) {};
				\draw[redarrow, line width=1mm] (corrw)--(corrwin);
			\end{tikzpicture}
		\end{adjustbox}
		\caption{100 pixel shifted images.} \label{fig:shiftedcat}
	\end{subfigure}
	\vskip\baselineskip
	\begin{subfigure}{\linewidth}
		\centering
		\begin{tikzpicture}
			\begin{axis}
				\addplot3 [surf, mesh/rows=19, mesh/ordering=x varies, shader=interp]
				table[col sep=comma] {figures/registration/cross-correlation/surf.csv};
				\node[above] at (axis cs:-110,-110,224141) {\((-100, -100)\)};
			\end{axis}
		\end{tikzpicture}
		\caption{Cross-correlation for various \(\Delta x, \Delta y\) shifts, with highest correlation at the shift.}
		\label{fig:crosscorr}
	\end{subfigure}
	\vskip\baselineskip
	\begin{subfigure}{\linewidth}
		\centering
		\begin{tikzpicture}
			\begin{axis}
				\addplot3 [surf, mesh/rows=19, mesh/ordering=x varies, shader=interp]
				table[col sep=comma] {figures/registration/cross-correlation/delta_surf.csv};
				\node[above] at (axis cs:-110,-110,1) {\((-100, -100)\)};
			\end{axis}
		\end{tikzpicture}
		\caption{Phase-correlation for various \(\Delta x, \Delta y\) shifts, with single peak at the correct shift.}
		\label{fig:phasecorr}
	\end{subfigure}
	\caption{Cross-correlation feature matching.}
	\label{fig:nccshift}
\end{figure}

%
The NCC image registration technique performs a grid-search over possible shifts and compute the cross-correlation of the shifted images (see figure~\ref{fig:shiftedcat}) (usually over a fixed cross-correlation window rather than the entire image for the sake of computational efficiency).
%
The maximum response as a function \(\Delta x, \Delta y\) is the imputed translation between the images (see figure~\ref{fig:crosscorr}).
%
An alternative but closely related method is \textit{phase correlation}, based on the Fourier shift theorem\anote{fouriershift}  (the value of going to Fourier space is availability of highly optimized algorithms for computing the Fourier transform).
%
Let \(\mathcal{X}(u,v) = \mathcal{F}\{X(x,y)\}, \; \mathcal{Y}(u,v) = \mathcal{F}\{Y(x,y)\}\) be the Fourier transforms of the displaced images.
%
Then
\[
    \mathcal{X} = \mathcal{Y}  e^{-2 \pi i (\frac{u \Delta x}{M} + \frac{v \Delta y}{N})}
\]
where \(N,M\) are the dimensions of the images
%
and the \textit{normalized cross-power spectrum}
\begin{align}
    R(u,v) & = \frac{\mathcal{X}\circ \mathcal{Y}^{*}}{|\mathcal{X} ||\mathcal{Y}^{*} |} \nonumber                                                                                                                            \\
           & = \frac{\mathcal{X}\circ \mathcal{X}^{*} \, e^{2 \pi i (\frac{u \Delta x}{M} + \frac{v \Delta y}{N})}}{|\mathcal{X} ||\mathcal{X}^{*} \, e^{-2 \pi i (\frac{u \Delta x}{M} + \frac{v \Delta y}{N})} |} \nonumber \\
           & = \frac{\mathcal{X}\circ \mathcal{X}^{*} \, e^{2 \pi i (\frac{u \Delta x}{M} + \frac{v \Delta y}{N})}}{|\mathcal{X} ||\mathcal{X}^{*} |} \nonumber                                                               \\
           & = e^{2 \pi i (\frac{u \Delta x}{M} + \frac{v \Delta y}{N})} \label{eqn:singleexp}
\end{align}
where \(\circ\) is Hadamard product\anote{hadamard}, \(|\mathcal{X}|\) is the magnitude of \(\mathcal{X}\), \(\mathcal{X}^*\) is the complex conjugate of \(\mathcal{X}\) and we've used the fact that \(|e^{iz}|=1\) for all \(z\).
%
Then the inverse Fourier transform of eqn.~\eqref{eqn:singleexp}
\[
    \mathcal{F}^{-1}\left\{ R(u,v) \right\} = r(x,y) = \delta(x + \Delta x, y + \Delta y)
\]
is a single peak (see figure~\ref{fig:phasecorr}) at \((\Delta x, \Delta y)\).
%
The simplest implementation of NCC only identifies translations but it can be extended to affine transforms\cite{berthilsson1998}.

\subsubsection{Transform Estimation}
Furthermore, the transformation to be estimated should incorporate prior knowledge about the motion model but simultaneously lead to a tractable estimation problem (i.e. reasonable number of parameters).
%In order to compute scale invariant features a multi-tier pyramid of images at progressively smaller resolutions is constructed.
%%
%Each tier has images at half the resolution of the previous, with the first tier in fact being a 2x upscaling of the original image.
%%
%Gaussian smoothing, at uniformly increasing \(\sigma_0, k \sigma_0, k^2 \sigma_0, \mathellipsis\) scale factors, is then applied to the images (see figure~\ref{fig:sift_pyramid}):
%\begin{equation}
%    L(x,y,k^n\sigma_0) = G(x,y,k^n\sigma_0) \ast X(x,y)
%\end{equation}
%where \(G(x,y,k^n\sigma)\) is a Gaussian filter.
%
%With the aim being to compute image curvature extrema, the Difference of Gaussian is computed as an approximation of Laplace of Gaussians\anote{log}:
%\begin{equation}
%    D(x,y,k^{n+1}\sigma_0) = L(x,y,k^{n+1}\sigma_0) - L(x,y,k^n\sigma_0)
%\end{equation}
%This pyramid is what is used to identify extrema; an extremum is a maximum or minimum in its 26-pixel scale-space neighborhood.
%%
%The intensity of the point in the differenced image is compared to its neighbors in \(3 \times 3\) windows above and below in scale, and adjacent in space (see figure~\ref{fig:sift_scale}).
%%
%Unfortunately this produces too many candidate keypoints; the next step in the algorithm is to fit a quadratic in the neighborhood of a candidate keypoint \(\bm{c} = (x_c, y_c, k^c\sigma_0)\):
%\begin{equation}
%    D(\bx) \approx D(\bm{c}) + \nabla D(\bm{c})^T \bx + \frac{1}{2}\bx^T H_{D}(\bm{c}) \bx
%    \label{eqn:dogapprox}
%\end{equation}
%where \(H_D\) is the Hessian of \(D\).
%%
%Then the true local extremum is the point \(\hat{\bm{x}}\) for which the derivative of eqn.~\eqref{eqn:dogapprox} is zero, i.e.
%\begin{equation}
%    \hat{\bm{x}} = -H_{D}^{-1}(\bm{c}) \nabla D(\bm{c})
%\end{equation}
%Low contrast points are rejected if \(|D(\hat{\bx})| < \tau\) a cut-off threshold.
%%
%Finally keypoints on edges of small principle curvature\anote{smallcurvature} are rejected.
%
%Once stable keypoints are identified, characterized by their coordinates and scale, an orientation is assigned to them in order that they function as rotation invariant features.
%


%
%For a static scene and an imaging system with 6 degrees of freedom, motion is dependent on the geometry of the scene and potentially complex (due to occlusion and parallax).
%%
%This pertains to image registration where we seek to relate \(\Xkone\) to \(\Xk\):
%\begin{equation*}
%    \Xkone(x,y) = \Xk(x + v_x(x,y), y + v_y(x,y))
%\end{equation*}
%%
%For small motions we can approximate \(\Xk\) by its first order Taylor series:
%\begin{align}
%    \Xkone(x,y) &= \Xk(x + v_x(x,y), y + v_y(x,y)) \\
%    &\approx \Xk(x,y) + v_x(x,y)\frac{\partial \Xk}{\partial x} + v_y(x,y)\frac{\partial \Xk}{\partial y}\label{eqn:motiontaylor}
%\end{align}
%Evaluating equation~\ref{eqn:motiontaylor} at every pixel gives a set of linear equations that enable us to fit one of the models in table~\ref{table:transformations}.
%%
%We focus on affine motion primarily because it is easy to estimate and secondarily because the composition of multiple affine transformations is an affine transformation (enabling us to register more than 2 images by building up the necessary transformations incrementally).
%%
%In this instance it can be seen that at \(\bx' \coloneqq (x', y') = (x + v_x(x,y), y + v_y(x,y))\) in \(\Xkone\) are related to pixels at \(\bx \coloneqq (x,y)\) in \(\Xk\) by a translation and a rotation:
%\begin{equation}
%    \bx' = R \bx + \bm{t}
%\end{equation}
%%
%Note that for nonstatic scenes the registration problem becomes "exponentially" more difficult as many more parameters need to be estimated.
%%
%Furthermore registration and super resolution are not independent since the data being used to estimate the registration transforms is blurry and noisy; to wit perfectly resolved images could be much more effectively registered.
%
\subsection{Gaussian Process}\label{subsec:gaussianprocess}
%\begin{table}
    \begin{center}
        \begin{tabular}[t]{@{}lp{2.6cm}}

            Model & When Applicable \\
            \\
            \hline
            \\
            {$\begin{aligned}[t]
                  v_x(x,y) &= p_1 x + p_2 y + p_3 \\
                  v_y(x,y) &= p_4 x + p_5 y + p_6
            \end{aligned}$} & Planar scene with orthographic projection \\
            \\
            {$\begin{aligned}[t]
                  v_x(x,y) &= \frac{p_1+p_2 x + p_3 y}{p_7 + p_8 x + p_9 y} - x\\
                  v_y(x,y) &= \frac{p_4+p_5 x + p_6 y}{p_7 + p_8 x + p_9 y} - y\\
            \end{aligned}$} & Planar scene with full prospective projection \\
            \\
            {$\begin{aligned}[t]
                  v_x(x,y) &= \omega_Z y + \frac{\omega_X xy}{l} - \frac{\omega_Y x^2}{l} - \omega_Y l \\
                  &\approx p_1 y + p_2 xy + p_3 x^2 + p_4 \\
                  v_y(x,y) &= -\omega_Z x - \frac{\omega_Y xy}{l} + \frac{\omega_X y^2}{l} + \omega_X l \\
                  &\approx p_5 y + p_6 xy + p_7 x^2 + p_8 \\
                  &
            \end{aligned}\(} & Approximate for prospective projection with only \)\omega_X, \omega_Y, \omega_Z\( euler angle rotations (\)l$ is focal length) \\
            \\
            {$\begin{aligned}[t]
                  v_x(x,y) &\approx p_1 x + p_2 y + p_3 x^2 + p_4 xy + p_5 \\
                  v_y(x,y) &\approx p_6 x + p_7 y + p_8 y^2 + p_9 xy + p_{10} \\
                  &
            \end{aligned}$} & Approximate for planar scene with full prospective projection\\
            \\
        \end{tabular}
    \end{center}
    \caption{Motion models \cite{Trucco:1998:ITC:551277}. Note \(x,y\) are pixel coordinates and \(p_i\) are parameters that need to be estimated.}
    \label{table:transformations}
\end{table}



\section{Classical Algorithms}\label{sec:classical-algorithms}
\localtableofcontents
\begin{figure}
    \centering
        \forestset{
          direction switch/.style={
            where level>=1{folder, grow'=0}{for children=forked edge},
          },
        }
        \begin{forest}
          direction switch
          [SR
            [MISR
              [Interpolation
                [Adaptive Filtering]
              ]
              [Statistical
                [Back Projection]
                [Markov Random Fields]
                [Bilateral Total Variation]
              ]
            ]
            [SISR
                [Manifold Learning]
                [Compressed Sensing]
                [Projection]
                [Belief Networks]
            ]
          ]
        \end{forest}
    \caption{Taxonomy of classical SR techniques}
    \label{fig:taxonomy}
\end{figure}

Figure~\ref{fig:taxonomy} lays out a rough taxonomy of classical SR algorithms. We cover algorithms from each "genus" and others that don't neatly fit into the taxonomy.
\subsection{Interpolation}
\subsection{Estimation}
\subsection{Example based}


%\section{Deep Learning Algorithms}\label{sec:deep-learning-algorithms}
%\localtableofcontents
%\input{deep.tex}
%\section{Future Research}\label{sec:future-research}
%\section{Conclusion}\label{sec:conclusion}
%\section{Appendix}\label{sec:appendix}
%TODO: work out diffraction circular aperture
%
%TODO: workout poisson noise
%
%TODO: workout conjugate gradients
%
%TODO: workout belief propagation
%
%\section*{Acknowledgments}
\newpage
\printbibliography

\end{document}



