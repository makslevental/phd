\documentclass[journal]{IEEEtran}
%\usepackage[caption=false]{subfig}
\usepackage{amssymb}
%\usepackage{textcomp}
%\usepackage[pdftex]{graphicx}
\usepackage[hyperfootnotes=false]{hyperref}\usepackage{amsmath}
\usepackage[backend=bibtex, style=ieee]{biblatex}
\addbibresource{quals.bib}
% Note that the amsmath package sets \interdisplaylinepenalty to 10000
% thus preventing page breaks from occurring within multiline equations. Use:
\interdisplaylinepenalty=2500
%\hyphenation{op-tical net-works semi-conduc-tor}

\begin{document}
    \title{Super Resolution for Automated Target Recognition}
    \author{Maksim Levental}
    \maketitle

    \begin{abstract}
        Super resolution is the process of producing high-resolution images from low-resolution
        images while preserving ground truth about the subject matter of the images and potentially
        inferring more such truth. Algorithms that successfully carry out
        such a process are broadly useful in all circumstances where HR imagery is either difficult or impossible
        to obtain. In particular we look towards super resolving images collected using longwave infrared (LWIR)
        cameras; high resolution sensors for such cameras do not currently exist. We present an exposition of
        motivations and concepts of super resolution in general and current techniques, with a qualitative comparison of
        such techniques. Finally we suggest directions for future research, in particular with applications to LWIR images.
    \end{abstract}

    \section{Introduction}\label{sec:introduction}

%    what is super resolution?
    For typical imaging use-cases high resolution images are preferable to low resolution images; higher resolutions are
    desirable in and of themselves (for purposes of human interpretation or automated analysis) and as inputs to later
    image processing transformations that can degrage image quality (by virtue of quantization or compression for example).
    %
    Here by resolution we mean in particular the ability of the imaging system to distinguish spatial features in
    the image plane rather than the closely related notions of spectral resolution (ability to distinguish frequencies)
    or temporal resolution (ability to distinguish motions). In theory the resolving power of an imaging system is
    primarily determined by the number of independent sensor elements that comprise that imaging system and each of
    which collects a component of the ultimate image; naturally then a way to increase the resolution of such a system
    is to increase the density of such sensor elements
    per unit area. Unfortunately, and counterintuitively, since the number of photons incident on each sensor decreases
    as the sensor shrinks, Poisson noise\footnote{Otherwise known as shot-noise, this is the result of incoming photons
    being distributed $\sim\text{Pois}(\lambda)$ and hence with $\text{SNR} = \sqrt{\lambda}$.}. thwarts that idea.
    Furthermore, while sensor density is primary, secondary effects due to physical optics limit spatial resolution as well;
    the point spread of a lens system, i.e. the impulse response of the system, chromatic aberrations (differing indices of
    refraction for differing wavelengths of light), and motion blur all function to blur details of the image plane.

    Super resolution (SR) is a collection of methods\footnote{We will often use the verb form "to super resolve"
    in order to denote the use of one or more such methods.} that augment the resolving power of an imaging system.


    TherefPractically speaking, there exist hardware and software solutions for increasing the resolution of an imaging
    system, but owing to a "Ship of Theseus" consideration, we discount such propositions as, for example, increasing
    the number of pixels per unit area on the imaging sensor, and focus exclusively on software techniques.

%    why is it useful?

%    which domains is it useful in?

%    how is it useful in those domains?

%    not only useful for humans but also machines.


    The benefit of enhancing images using super resolution techniques includes not only more pleasing or more readily interpretable images
    for human consumption but higher quality features for automated learning systems as well.
    In particular object detection systems trained on super-resolved images outperform those trained on the low
    resolution originals\cite{effectssuperres}. In domains such as satellite/aerial photography, medical imaging, and facial recognition
    high-resolution reconstruction of low-resolution samples is eminently useful - in that ab-initio acquisition of
    high-resolution images is either logistically difficult or impossible due to image apparatus limitations. For example in the
    instance of satellite imagery, acquisition of high-resolution imagery is primarily hampered by optics and
    physics\footnote{The angular resolution $R$ of a telescope with optical diameter $D = 2.4m$ observing visible light
    ($\sim500nm$) is approximately\cite{doi:10.1080.14786447908639684} \[R  \approx 1.220\frac{\lambda}{D} = 1.220 \frac{500\text{nm}}{2.4\text{m}} \approx 0.06 \text{arcsec}\]
    From an altitude of 250 km this corresponds to a ground sample distance of 6cm. This "loss of resolution" is further
    exacerbated by diffraction through the atmosphere.}. In contrast, in the cases of medical imaging
    (where procedures are invasive and patient exposure time needs to be minimized\cite{doi:10.1002.cmr.a.21249}) and
    facial recognition (e.g. for purposes of surveillance) the primary challenge is logistics and access to repeat collection
    opportunities.

    Of particular interest are situations for which imaging sensors

    The typical imaging system today consists of either a semiconductor integrated circuit, of either the
    charge-coupled device (CCD) or complementary metal-oxide-semiconductor (CMOS) type. The pheno




    \section{Background}\label{sec:background}
    \section{Classical Algorithms}\label{sec:classical-algorithms}
    \section{Deep Learning Algorithms}\label{sec:deep-learning-algorithms}
    \section{Future Research}\label{sec:future-research}
    \section{Conclusion}\label{sec:conclusion}

    \section*{Acknowledgments}
    \newpage
    \printbibliography

\end{document}



