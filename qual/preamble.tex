\usepackage{etoolbox}
\usepackage{amsmath}
\usepackage{widetext}

\usepackage[textwidth=7in,textheight=9.5in]{geometry} 
\usepackage{amsthm}
\usepackage{imakeidx}
\usepackage{transparent}
\usepackage[style=2,everyline=true,footnoteinside=true]{mdframed}
\usepackage{tabularx,booktabs}
\usepackage{commath}
\usepackage{esint} % various fancy integral symbols
\usepackage{tocloft}
\usepackage{etoc}
\usepackage{bm}
\usepackage{wrapfig}
\usepackage{multicol}
\usepackage{sepfootnotes}
\usepackage{bigfoot}
\usepackage{xparse}
\usepackage[bottom]{footmisc}
\usepackage{amsfonts}
\usepackage[T1]{fontenc}
\usepackage[pdftex]{graphicx}
\usepackage{tikz}
\usepackage{pgfplotstable}
\usepgfplotslibrary{patchplots}
\usepackage{xcolor}
\usepackage{siunitx}
\usetikzlibrary{
	patterns,
	chains,
	backgrounds,
	calc,
	shadings,
	shapes.arrows,
	arrows,
	shapes.symbols,
	shadows,
	positioning,
	decorations.markings,
	backgrounds,
	arrows.meta,
	external
}

\usetikzlibrary{matrix}
\usetikzlibrary{fit}
\usetikzlibrary{shapes.geometric}
\usepackage{tikz-3dplot}
% \tikzexternalize[mode=list and make]
% \tikzexternalize
\usepackage{bbm}
\usepackage{relsize}
\usepackage{pgfplots}
\pgfplotsset{compat=newest}
\usepackage[edges]{forest}
\usepackage{adjustbox}
\usepackage{mathtools}
\usepackage{physics}
\usepackage{mleftright}
\NewDocumentCommand{\evalat}{sO{\big}mm}{%
  \IfBooleanTF{#1}
   {\mleft. #3 \mright|_{#4}}
   {#3#2|_{#4}}%
}
\usepackage{breqn}
\usepackage{subcaption}
\DeclareCaptionFormat{subfig}{#1#2#3}
\DeclareCaptionSubType{figure}
\captionsetup[subfigure]{format=subfig,labelsep=space,labelformat=parens}


% \usepackage[backend=bibtex, bibencoding=utf8, style=ieee]{biblatex}
\usepackage[backend=biber, bibencoding=utf8, style=ieee, backref=true]{biblatex}
\usepackage[hyperfootnotes=false,hyperindex=true]{hyperref}
\usepackage[utf8]{inputenc}
\usepackage{svg}
\makeindex

\setlength{\columnsep}{1cm}

\setlength{\cftsubsecindent}{0pt}% Remove indent for \subsection
\setlength{\cftsubsubsecindent}{10pt}% Remove indent for \subsubsection
\setlength{\cftparaindent}{20pt}% Remove indent for \paragraph
\setlength{\cftsubparaindent}{30pt}% Remove indent for \paragraph

\setcounter{secnumdepth}{5}
\setcounter{tocdepth}{5}
\renewcommand\thesection{\arabic{section}}
% \renewcommand\thesectiondis{\arabic{section}}

\renewcommand\thesubsection{\thesection.\arabic{subsection}}
% \renewcommand\thesubsectiondis{\thesectiondis.\arabic{subsection}}

\renewcommand\thesubsubsection{\thesubsection.\arabic{subsubsection}}
% \renewcommand\thesubsubsectiondis{\thesubsectiondis.\arabic{subsubsection}}

\renewcommand{\theparagraph}{\thesubsubsection.\arabic{paragraph}}
% \renewcommand{\theparagraphdis}{\thesubsubsectiondis.\arabic{paragraph}}

%\makeatletter
%\newcounter{subparagraph}[paragraph]
%\newcommand\subparagraph{%
%\@startsection
%    {subparagraph}    % counter
%    {5}                              % level
%    {\z@}                    % indent
%    {0.7ex plus .5ex minus 0ex}     % afterskip {0ex}
%    {0.7ex plus .5ex minus 0ex}     % afterskip {0ex}
%    {\normalfont\normalsize\itshape}
%}
%\makeatother
%\renewcommand{\thesubparagraph}{\theparagraph.\arabic{subparagraph}}
%\def\thesubparagraphdis{\theparagraph.\arabic{subparagraph}}

\makeatletter
\patchcmd{\@makecaption}
{\\}
{.\ }
{}
{}
\makeatother

\def\tablename{Table}
\renewcommand{\thetable}{\arabic{table}}
\newcommand{\ue}[1]{
	\begin{tikzpicture}[#1]
		\draw[fill = black] (.25ex,.25ex) circle (.3ex);
		\draw[thick] (.55ex,.25ex) -- (1.55ex,.25ex);
		\draw[fill = black] (1.85ex, .25ex) circle (.3ex);
	\end{tikzpicture}%
}

\newcommand{\newterm}[1]{\textit{#1}\index{#1}}

\newfootnotes{a}
\hypersetup{
	colorlinks,
	linkcolor={red!50!black},
	citecolor={blue!50!black},
	urlcolor={blue!80!black}
}
\addbibresource{quals.bib}
\definecolor{darkblue}{HTML}{1f4e79}
\definecolor{lightblue}{HTML}{00b0f0}
\definecolor{salmon}{HTML}{ff9c6b}

% Note that the amsmath package sets \interdisplaylinepenalty to 10000
% thus preventing page breaks from occurring within multiline equations. Use:
\interdisplaylinepenalty=0
\interfootnotelinepenalty=10000
\raggedbottom

% fix footnote rule from ieee
\def\footnoterule{
	\relax
	\kern-5pt
	\hbox to \columnwidth{\hfill\vrule width \columnwidth height 0.4pt}
	\kern4.6pt
}

\newcommand{\Xk}{X_k}
\newcommand{\Xkone}{X_{k+1}}
\newcommand{\bx}{\bm{x}}
\newcommand{\bX}{\bm{X}}
\newcommand{\bxi}{\bm{x}_i}
\newcommand{\delx}{\bx - \bxi}
\newcommand{\by}{\bm{y}}
\newcommand{\bY}{\bm{Y}}
\newcommand{\byi}{\bm{y}_i}
\newcommand{\dely}{\by - \byi}
\newcommand{\zbx}{Z(\bx)}
\newcommand{\zbxi}{Z(\bxi)}
\newcommand{\bb}{\bm{\beta}}
\newcommand{\hzbx}{\hat{Z}(\bx)}
\newcommand{\etal}{\textit{et al.}~}


\makeatletter
\renewcommand\subsubsection{%
	\@startsection
	{subsubsection}                 % type
	{3}                             % level
	{\z@}                    % indent
	%{3.5ex plus 1.5ex minus 1.5ex}  % beforeskip {0ex plus 0.1ex minus 0.1ex}
	{0.7ex plus .5ex minus 0ex}     % afterskip {0ex}
	{0.7ex plus .5ex minus 0ex}     % afterskip {0ex}
	{\normalfont\normalsize\itshape}% style
}
\makeatother
\makeatletter
\renewcommand\paragraph{%
	\@startsection
	{paragraph}                 % type
	{4}                             % level
	{\z@}                    % indent
	{0.7ex plus .5ex minus 0ex}     % afterskip {0ex}
	{0.7ex plus .5ex minus 0ex}     % afterskip {0ex}
	{\normalfont\normalsize\itshape}% style
}
\makeatother

\newlength\tocrulewidth
\setlength{\tocrulewidth}{1.5pt}
%\etocsettocstyle{\rule{\linewidth}{\tocrulewidth}\vskip0.5\baselineskip}{\rule{\linewidth}{\tocrulewidth}}
%\etocsettocstyle{\rule{.5\linewidth}{\tocrulewidth}\vskip0.5\baselineskip}{\rule{.5\linewidth}{\tocrulewidth}}
\etocsettocstyle{\medskip\hrule\medskip}{\medskip\hrule\medskip}
\etocsettocdepth{5}
\tikzset{
		greenarrow/.style={
				-latex,black!60!green,thick,solid
			},
		redarrow/.style={
				-latex,red,thick,solid
			},
		bluearrow/.style={
				-latex,blue,thick,solid
			},
		dot/.style = {
				circle,
				fill,
				minimum size=#1,
				inner sep=0pt,
				outer sep=0pt
			},
		dot/.default = 6pt
}

% colors
\definecolor{blue}{RGB}{38,139,210}
\definecolor{cyan}{RGB}{42,161,152}
\definecolor{violet}{RGB}{108,113,196}
\definecolor{red}{RGB}{220,50,47}
\definecolor{base01}{RGB}{88,110,117}
\definecolor{base02}{RGB}{7,54,66}
\definecolor{base03}{RGB}{0,43,54}

\definecolor{snowymint}{HTML}{E3F8D1}
\definecolor{wepeep}{HTML}{FAD2D2}
\definecolor{portafino}{HTML}{F5EE9D}
\definecolor{plum}{HTML}{DCACEF}
\definecolor{sail}{HTML}{A3CEEE}
\definecolor{highland}{HTML}{6D885A}

\tikzstyle{signal}=[arrows={-latex},draw=black,line width=1.5pt,rounded corners=4pt]

% RNN
\tikzstyle{block}=[draw=black,line width=1.0pt]
\tikzstyle{cell}=[style=block,draw=highland,fill=snowymint,
    rounded corners]
\tikzstyle{celllayer}=[style=block,draw,fill=portafino,
    inner sep=1pt,outer sep=0,
    minimum width=28pt, minimum height=14pt]
\tikzstyle{pointwise}=[style=block,ellipse,fill=wepeep,
    inner sep=1pt,outer sep=0, minimum size=12pt]

\def\iolen{24pt}
\def\intergape{2pt}

% MLP and CNN
\tikzstyle{netnode}=[circle, inner sep=0pt, text width=22pt, align=center, line width=1.0pt]
\tikzstyle{inputnode}=[netnode, fill=sail,draw=black]
\tikzstyle{hiddennode}=[netnode, fill=snowymint,draw=black]
\tikzstyle{outputnode}=[netnode, fill=plum,draw=black]

% Architecture
\def\layerwidth{90pt}
\def\layerheight{14pt}

\tikzstyle{layer}=[style=block, draw, fill=black!20!white,
    inner sep=1pt,outer sep=0, font=\footnotesize,
    text centered, 
    minimum width=\layerwidth, minimum height=\layerheight]

\tikzstyle{fc}=[style=layer, fill=blue!30!white]
\tikzstyle{conv}=[style=layer, fill=green!30!white]
\tikzstyle{activation}=[style=layer, fill=orange!30!white]
\tikzstyle{pool}=[style=layer, fill=red!30!white]
\tikzstyle{bn}=[style=layer, fill=cyan!30!white]
\tikzstyle{recurrent}=[style=layer, fill=purple!30!white]
\tikzstyle{softmax}=[style=layer, fill=yellow!30!white]
\tikzstyle{point}=[]
\tikzstyle{branch}=[coordinate]

\def\vlayerwidth{30pt}
\def\vlayerheight{3pt}
\def\vblockheight{28pt}

\tikzstyle{vlayer}=[minimum width=\vlayerwidth, minimum height=\vlayerheight]
\tikzstyle{vblock}=[minimum width=\vlayerwidth, minimum height=\vblockheight, text width=1cm, align=center]


% Precision, Recall
\colorlet{fn}{gray!90!green!30!white}
\colorlet{tp}{green!40!white}
\colorlet{fp}{red!40!white}
\colorlet{tn}{gray!90!red!20!white}

\newcommand{\iu}{\mathrm{i}\mkern1mu}